
\section{Índice de Sustentabilidade}

Revela o estado de um sistema ou fenômeno, é uma síntese das características
ou variáveis analisadas. Um índice pode ser construído para analisar
dados através da junção de um jogo de elementos com relacionamentos
estabelecidos. Entende-se o termo índice como um valor numérico que
representa a correta interpretação da realidade de um sistema simples
ou complexo (natural, econômico ou social), utilizando, em seu cálculo,
bases científicas e métodos adequados. O índice pode servir como um
instrumento de tomada de decisão e previsão, e é considerado um nível
superior da junção de um jogo de indicadores ou variáveis \citep{SicheAgostinho2007}

No projeto SustenAgro os índices serão dados numéricos gerais representam
a soma do estado de cada indicador em cada dimensão e atributo norteador,
cada indicador pode o valor de mais um ou menos um (+1 -1), que permitira
quantificar a sustentabilidade em cada aspecto do sistema produtivo
e fazer comparações com outros sistemas produtivos compatíveis.

\section{Limiares de Sustentabilidade}

Os limiares são os pontos mínimo e máximo aceitáveis na amplitude
da sustentabilidade para cada indicador.

Considerando que a sustentabilidade permanece sempre no futuro \citep{gliessman2001agroecologia},
dado o compromisso que os sistemas têm de garantir as necessidades
das gerações futuras, a sustentabilidade será considerada como algo
relativo no espaço e no tempo, ou seja, um sistema pode ser mais ou
menos sustentável do que outro.

Esta representação será realizada pelos limiares de sustentabilidade
que poderão variar de acordo com o sistema de produção considerado
e, principalmente, deve variar de modo a representar com propriedade
das especificidades regionais e microrregionais.

Dentro de uma escala, devem ser estabelecidos limiares críticos, ou
seja, aqueles em que concordamos que determinada situação (característica,
produto, serviço) apesar de não ser totalmente sustentável possui
níveis de sustentabilidade aceitáveis para que a sustentabilidade
seja efetiva (verdadeira), apesar de não ser a ideal. O limiar é um
ponto que estabelece um limite, geralmente é o princípio, mas no nosso
caso, são os limites que apontam que determinada característica, produto,
ou serviço, está dentro do que for considerado sustentabilidade, serão
os pontos mínimo e máximo aceitáveis na amplitude da sustentabilidade.

Dentro desta escala, estabelecemos limiares críticos, ou seja, aqueles
em que concordamos que determinada situação (característica, produto,
serviço) apesar de não ser totalmente sustentável (nota máxima), possui
níveis de sustentabilidade aceitáveis para que a sustentabilidade
seja efetiva (verdadeira), apesar de não ser a ideal. Neste caso,
o limiar mínimo de sustentabilidade assumiria um valor variável.

Exemplo de limiar da sustentabilidade que poderão ser empregados pela
equipe do projeto:
\begin{itemize}
\item Nome do Indicador: Distância Usina / Área de Produção de cana
\item Descrição do indicador: usualmente, em tradicionais regiões produtoras
de cana utiliza-se de uma distância econômica padrão da produção de
50 quilômetros até a indústria. Esta distância é determinada pelos
altos custos de transporte da cana até a unidade industrial, sendo
um dos fatores decisivos na rentabilidade da lavoura (CNA/SENAR, 2007).
\item Limiares de sustentabilidade, teria dois estados possíveis 

\begin{itemize}
\item Distância de até 50 km: Mais sustentável (+1)
\item Distância de mais de 50 km: Menos sustentável (-1)
\end{itemize}
\end{itemize}
Baseando-se nesse conceito sobre limiares é possível desenhar metodologias
de avaliação onde sejam usados os valores numéricos de cada limiar
para fazer comparações, o que permite definir se determinado sistema
produtivo e/ou contexto é mais sustentável do que outro sistema produtivo
e/ou contexto.

\section{Indicadores de Sustentabilidade}

Os indicadores são instrumentos usados para avaliar uma determinada
realidade levando em conta variáveis pertinentes para sua composição.
Além da avaliação, o uso de indicadores permite medir e monitorar
aspectos da realidade. Ele agrega, quantifica e simplifica informações
sobre fenômenos complexos de modo que as tendências ficam mais significativas
e aparentes, a fim de melhorar o processo de entendimento e comunicação\citep{bossel1999indicators,van2005indicadores}.

De acordo com \citep{gallopin1996environmental} os melhores indicadores
são aqueles que simplificam as informações relevantes, tornando os
fenômenos mais claros. Como um indicador é utilizado para atingir
diversos objetivos, é necessário definir um requisito geral para selecionar
indicadores e validar a escolha. A finalidade de um indicador de sustentabilidade
é refletir as alterações nas propriedades fundamentais de um sistema
\citet{CaminoAndMuller1993} e advertir sobre eventuais perturbações
potenciais.\citep{ferraz2003}

Normalmente um indicador é utilizado como um pré-tratamento aos dados
originais \citet{SicheAgostinho2007}. Indicadores são parâmetros
que podem ser utilizados como medida do cumprimento dos critérios
\citep{moret2006criterios}. Deve-se observar que não é possível o
desenvolvimento de um indicador global, por isso é necessário buscar
no tempo a evolução da sustentabilidade dos sistemas \citep{CaminoAndMuller1993}.
Não há indicadores universais, pois eles podem variar segundo o problema
ou objetivo da análise.

Enquanto às características desejáveis para um bom indicador, deve-se
ter uma boa definição da fonte dos dados base para o levantamento,
possibilidade de calibração, possibilidade de comparação com critérios
legais ou outros padrões/metas existentes, facilidade e rapidez de
determinação e interpretação, grau de importância e validação científica,
sensibilidade do público-alvo, custo de implementação e possibilidade
de ser rapidamente atualizado. Nessa mesma linha, \citep{zampieri2003metodo}
baseado em vários autores, cita como requisitos para a seleção de
indicadores de avaliação de sustentabilidade: \renewcommand{\labelenumi}{\roman{enumi}.}
\begin{enumerate}
\item Serem mensuráveis quantitativa e qualitativamente, além de terem pertinência
ao objeto e à natureza do processo avaliado; 
\item Poder coletar as informações por baixo custo, ser de fácil execução
e apresentar dados cientificamente válidos; 
\item Serem concebidos para que o agricultor participe das medições, adaptados
às necessidades dos usuários da informação e estarem embasados em
linguagem clara; 
\item Serem sensíveis às mudanças do sistema ao detectar a magnitude dos
desvios e tendências, oferecendo prognósticos e perspectivas para
planejar e tomar decisões; 
\item Fornecerem indicação clara a respeito da sustentabilidade do sistema
estudado e refletirem os impactos estudados sob o enfoque integrado; 
\item Representarem padrões ecológicos, sociais, econômicos e espaciais,
que tenham correspondência e sensibilidade com o nível de agregação
do sistema considerado; 
\item Conter um nível de agregação que permita comparações individuais,
intertemporais e o cruzamento com outros indicadores; 
\item Fornecerem informações para avaliar os trade-offs entre as dimensões
da sustentabilidade e correlações com os processos dos ecossistemas; 
\item Poder ter repetibilidade, de modo que as medições possam ser realizadas
por diferentes pessoas e que os resultados sejam comparáveis
\item A construção do indicador deve observar parâmetros politicamente corretos.
\end{enumerate}
A OECD (1993) estabelece três requisitos para selecionar indicadores:
relevância política e utilidade para usuários, solidez analítica e
mensurabilidade. 

Alguns exemplos de indicadores levantados no desenvolvimento do método
SustenAgro são: 
\begin{enumerate}
\item Risco climático; 
\item Diversidade de culturas anuais; 
\item Tipo de solo; 
\item Risco de deficit hídrico; 
\item Produtividade da terra; 
\item Renovabilidade energética nos sistemas de produção; 
\item Balanço de nutrientes (nitrogênio e fósforo); 
\item Área de cultivo/áreas preservadas.
\end{enumerate}
Os indicadores do presente projeto são uma representação dos fatores
críticos que existem no sistema de produção de cana-de-açúcar no centro-sul
do Brasil em cada dimensão da sustentabilidade, pelo qual a metodologia
e o sistema SustenAgro é aplicável nesse contexto, no caso de quer
aplicar o sistema de avaliação da sustentabilidade em outro contexto
é necessário mudar os indicadores a cada contexto especifico. 

\section{Dados fornecidos pela Unidade de Meio Ambiente da Embrapa }

A principal fonte de dados para este projeto foi fornecida pela pesquisa
de \citet{oliveira:2013}, onde inicialmente foram identificados 62
indicadores de sustentabilidade no sistema de cana-de-açúcar, os quais
foram analisados e caracterizados, gerando 39 indicadores como os
mais relevantes \citet{BRUMATTI:2015}, por meio de uma validação
com porcentagem maior ou igual a 60\% feita por uma comunidade de
especialistas em sustentabilidade.

A seguintes tabelas mostram os indicadores resultantes, os quais foram
classificados nas três dimensões da sustentabilidade.

Os indicadores da tabela \ref{tab:Indicadores-de-sustentabilidade-ambiental}
representam os valores críticos da dimensão ambiental integrando fenômenos
do solo, dos recursos hídricos e climáticos, os quais permitem caracterizar,
quantificar e comparar o estado da dimensão ambiental de uma unidade
produtiva com outras.

\begin{table}[h]
\begin{tabular}{|>{\raggedright}p{14cm}|}
\hline 
\textbf{Indicadores da dimensão ambiental}\tabularnewline
\hline 
\hline 
Quantificação da erosão potencial segundo a Equação Universal de Perda
de Solo (USLE – \foreignlanguage{english}{Universal Soil Loss Equation})\tabularnewline
\hline 
Compactação do solo\tabularnewline
\hline 
Ocorrência de queimada de palha no campo\tabularnewline
\hline 
Emissão e suspensão de micropartículas (fuligem)\tabularnewline
\hline 
Localização geográfica da cultura em relação à aptidão agroclimática\tabularnewline
\hline 
Localização geográfica da cultura em relação à aptidão edáfica\tabularnewline
\hline 
Localização geográfica da cultura em relação à aptidão edafoclimática\tabularnewline
\hline 
Áreas de Preservação Permanente (APP) recuperadas/conservadas\tabularnewline
\hline 
Comprovação de averbação da área de Reserva Legal\tabularnewline
\hline 
Cumprimento com os Termos de Compromisso de Recuperação Ambiental\tabularnewline
\hline 
\end{tabular}\caption{Indicadores de sustentabilidade de SustenAgro na dimensão ambiental
\label{tab:Indicadores-de-sustentabilidade-ambiental}}
\end{table}

Os indicadores da tabela \ref{tab:Indicadores-de-sustentabilidade-social}
representam os valores críticos da dimensão social integrando fenômenos
de emprego, saúde e treinamento, os quais permitem caracterizar, quantificar
e comparar o estado da dimensão social de uma unidade produtiva com
outras.

\begin{table}[h]
\begin{tabular}{|>{\raggedright}p{14cm}|}
\hline 
\textbf{Indicadores da dimensão social}\tabularnewline
\hline 
\hline 
Poder de compra do trabalhador\tabularnewline
\hline 
\hline 
Taxa de formalidade do emprego\tabularnewline
\hline 
\hline 
Índice Parcial de Educação\tabularnewline
\hline 
\hline 
Índice de internações decorrentes de problemas respiratórios\tabularnewline
\hline 
\hline 
Registro de treinamentos, capacitação ou requalificação de trabalhadores\tabularnewline
\hline 
\end{tabular}

\caption{Indicadores de sustentabilidade de SustenAgro na dimensão social \label{tab:Indicadores-de-sustentabilidade-social} }
\end{table}

Os indicadores da tabela \ref{tab:Indicadores-de-sustentabilidade-economica}
representam os valores críticos da dimensão econômica integrando fenômenos
de emprego, saúde e treinamento, os quais permitem caracterizar, quantificar
e comparar o estado da dimensão social de uma unidade produtiva com
outras.

\begin{table}[h]
\begin{tabular}{|>{\raggedright}p{14cm}|}
\hline 
\textbf{Indicadores da dimensão econômica}\tabularnewline
\hline 
\hline 
\textbf{Indicadores Agrícola/Industrial}\tabularnewline
\hline 
Implantação de biorrefinarias\tabularnewline
\hline 
Rotação de cultura (soja)\tabularnewline
\hline 
Área planta/Área colhida\tabularnewline
\hline 
Atender à Norma Regulamentadora (NR-31)\tabularnewline
\hline 
Longevidade da cana\tabularnewline
\hline 
Distância usina/produção de cana\tabularnewline
\hline 
Controle de pragas favorecidas pela não-queima\tabularnewline
\hline 
Cana queimada manual\tabularnewline
\hline 
Adoção do plantio direto\tabularnewline
\hline 
Predominância da conversão de pastagem em cana-de-açúcar, do que outras
culturas/florestas em cana-de-açúcar\tabularnewline
\hline 
Ocorrência de reutilização de recursos hídricos\tabularnewline
\hline 
Condições favoráveis à mecanização\tabularnewline
\hline 
Otimização do transporte da cana\tabularnewline
\hline 
Consumo de diesel\tabularnewline
\hline 
Variedades melhoradas para condições eco regionais mais específicas\tabularnewline
\hline 
\tabularnewline
\hline 
\textbf{Indicadores Produtos/Subprodutos}\tabularnewline
\hline 
Relação preço gasolina/etanol\tabularnewline
\hline 
Inclusão do Etanol como Commodity\tabularnewline
\hline 
Adoção da tecnologia flex-fuel por outros países\tabularnewline
\hline 
Regulação de comércio de distribuição\tabularnewline
\hline 
Número de contrato para fornecer bioeletricidade\tabularnewline
\hline 
Infraestrutura para a produção de biocombustíveis de 2ª. e 3ª. gerações\tabularnewline
\hline 
\tabularnewline
\hline 
\textbf{Indicadores Tecnológicos}\tabularnewline
\hline 
Desenvolvimento de leveduras mais resistentes a concentrações elevadas
de álcool (Fermentação Extrativa)\tabularnewline
\hline 
\tabularnewline
\hline 
\textbf{Indicadores Políticos}\tabularnewline
\hline 
Iniciativas do poder público com a proteção ao ambiente\tabularnewline
\hline 
\end{tabular}

\caption{Indicadores de sustentabilidade de SustenAgro na dimensão econômica\label{tab:Indicadores-de-sustentabilidade-economica}}
\end{table}

Cada um dos anteriores indicadores foram definidos com um conjunto
de pelo menos um componente de indicador, estes componentes permitem
quantificar por meio de uma variável quantitativa o estado do indicador,
os quais estão definidos em termos do domínio que são de fácil interpretação
pelas pessoas relacionadas com sustentabilidade em agricultura.

\section{Considerações finais}

Os dados e especificações fornecidos pela Embrapa Meio Ambiente e
pela APTA conseguiram explicar o conceito de avaliação de sustentabilidade
segundo a visão da Embrapa Meio Ambiente, más a complexidade envolvida
requereu identificar um tipo de KOS que permita representar cada uns
dos conceitos necessários que compõem o processo de avaliação da sustentabilidade.

Dito KOS precisa ser flexível e de fácil uso para conseguir se adaptar
às mudanças do domínio, devido a que durante o processo de modelagem
avalia a coerência dos dados, permitindo assim melhorar as especificações
de dito domínio.
