Uma vez cadastrada a unidade produtiva/fazenda disponibiliza-se a
opção de criar nova avaliação, ação que vai gerar a tela da figura
9 que permite visualizar as variaveis de eficiencia e os indicadores
para que os usuários preencham cada uma segundo a realidade da unidade
produtiva em avaliação, cada indicador ou variavel de eficiencia tem
varias opções que estão ligadas a valores que quantificam a sustentabilidade,
esses valores estão definidos na ontologia da sustentabilidade e serao
os valores de ingreso para a formula que vai gerar os indices da sustentabilidade. 

\begin{figure}
\includegraphics[scale=0.5]{\string"/media/john/Data/Master Degree/Dissertation Document/figures/SustenAgro-scenario\string".eps}

\caption{Cadastro das variaveis/indicadores}
\end{figure}

A partir desses desses dados cadastrados são gerados os resultados
do sistema que consistem na planilha de ediciência e custo, na planilha
da sustentabilidade e o relatorio do sistema, as planilhas permitem
a visualizar os atributos das variaveis de eficiência e dos indicadores
e a tela de relatorio que apresenta a matrix de avaliação onde são
relaciondas as variaveis de eficiência e de sustentabilidade, o relatio
é apresentado na figura 10. 

\begin{figure}
\includegraphics{\string"/media/john/Data/Master Degree/Dissertation Document/figures/SustenAgro-results\string".eps}

\caption{Formato das planilhas de resultado}
\end{figure}

