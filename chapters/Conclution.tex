Os resultados obtidos são: 
\begin{enumerate}
\item Ontologia em formatos da web semântica (RDF/OWL) da avaliação da sustentabilidade
no sistema de cana-de-açúcar. 
\item Protótipo de ontologia de interfaces gráficas em formatos da web semântica
(RDF/OWL) 
\item Linguagem de domínio especifico DSL, Decisioner, para definir e permitir
administrar os parâmetros e processos do sistema SustenAgro 
\item Protótipo de sistema de geração de interfaces suportado nas tecnologias
da web semântica 
\item Formulários para recolha de dados sobre sustentabilidade em cana-de-açúcar,
e o processo de colheita dos dados de algumas usinas do estado de
São Paulo 
\item Protótipo do sistema web Sustenagro que integra as ontologias e a
DSL, fornecendo um comportamento configurável em tempo de execução
dos parâmetros, processos e das interfaces gráficas de usuário
\end{enumerate}
Uma das finalidades deste projeto é construir um gerador de sistemas
de apoio a decisão que consiga suportar outros domínios de conhecimento,
propondo para a comunidade uma metodologia de desenvolvimento e manutenção
que fique simples para os usuários finais e assim permitir que os
especialistas no domínio façam as mudanças sem precisas dos especialistas
de T.I.

\section{Trabalhos Futuros}

Este capítulo apresenta os resultados estão uma versão da ontologia
de domínio do SustenAgro e artefatos para o desenvolvimento da interface
visual do sistema: User Stories, Scenarios, Story Boards, Mockups
e um protótipo para a interface do SustenAgro.

Os resultados obtidos são: 
\begin{enumerate}
\item Ontologia em formatos da web semântica (RDF/OWL) da avaliação da sustentabilidade
no sistema de cana-de-açúcar. 
\item Protótipo de ontologia de interfaces gráficas em formatos da web semântica
(RDF/OWL) 
\item Linguagem de domínio especifico DSL, Decisioner, para definir e permitir
administrar os parâmetros e processos do sistema SustenAgro 
\item Protótipo de sistema de geração de interfaces suportado nas tecnologias
da web semântica 
\item Formulários para recolha de dados sobre sustentabilidade em cana-de-açúcar,
e o processo de colheita dos dados de algumas usinas do estado de
São Paulo 
\item Protótipo do sistema web Sustenagro que integra as ontologias e a
DSL, fornecendo um comportamento configurável em tempo de execução
dos parâmetros, processos e das interfaces gráficas de usuário
\end{enumerate}
Uma das finalidades deste projeto é construir um gerador de sistemas
de apoio a decisão que consiga suportar outros domínios de conhecimento,
propondo para a comunidade uma metodologia de desenvolvimento e manutenção
que fique simples para os usuários finais e assim permitir que os
especialistas no domínio façam as mudanças sem precisas dos especialistas
de T.I.
