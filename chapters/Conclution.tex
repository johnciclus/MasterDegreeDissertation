As principais contribuições desta pesquisa foram o método e ferramenta
de geração de SADs usando ontologias e \foreignlanguage{english}{DSLs},
Esse método permite complementar o conhecimento descrito em uma ontologia
com uma DSL, que define o comportamento e formato de um SAD. 

Ontologia e DSL estão abertas a edição pelos especialistas de domínio.
Eles podem editá-las, através de editores online (na própria ferramenta),
e modificar aspectos fundamentais do SAD em tempo real. Essa característica
permite um ciclo de desenvolvimento mais curto e uma participação
ativa dos especialistas de domínio. Como o Framework Decisioner fornece
uma aplicação semiacabada (como todo o framework) que os próprios
especialistas de domínio podem modificar, isso tende a diminuir os
custos associados a criação de SADs. 

O Framework Decisioner, o SAD SustenAgro e o método de definição de
SADs proposto tiveram uma avaliação positiva por parte dos especialistas
do domínio sustentabilidade e usuários finais da Embrapa e do projeto
SustenAgro. Essas avaliações demostraram que as ferramentas e o método
proposto solucionaram, para o caso do SustenAgro, os problemas identificados
na pesquisa e que os objetivos específicos desta pesquisa foram satisfeitos.

Se deve ressaltar que, apesar dos bons resultados das avaliações e
do entusiasmo dos especialistas da Embrapa (que nunca tiveram tantas
facilidades para modificar o código de um projeto de SAD), ainda não
é possível, para especialistas de domínio, desenvolver um SAD, usando
o Decisioner, a partir do zero. Mas os resultados mostraram que eles
podem modificar muitos aspectos do sistema sem a ajuda de programadores.
O que é um avanço muito grande, se comparado com o que é possível
fazer em métodos de desenvolvimento tradicionais.

Outra importante contribuição foi a geração de uma versão funcional
do SAD SustenAgro. Ela permitiu cumprir os objetivos do projeto SustenAgro
da Embrapa, fornecendo uma ferramenta de avaliação da sustentabilidade
em cana-de-açúcar que pode efetivamente ser usada por empregados de
fazendas e usinas.

Para continuar a pesquisa, foram sugeridos trabalhos futuros. Dentre
os mais relevantes estão a metodologia de criação de ontologias e
as ferramentas web para editá-las.

\section{Resultados}

Os resultados obtidos são: 
\begin{enumerate}
\item Ontologia SustenAgro sobre avaliação de sustentabilidade em sistemas
produtivos de cana-de-açúcar na região centro-sul do Brasil. Ela representa
os principais conceitos desse domínio, necessários para o suporte
da avaliação de sustentabilidade. Dentre eles estão os indicadores,
unidades produtivas, microrregiões, índices e métodos de avaliação.
A ontologia padroniza o conhecimento dos especialistas em um formato
computável. Ela foi desenvolvida em parceria com os especialistas
de domínio de sustentabilidade da Embrapa.
\item Ontologia sobre tipos de dados e controles visuais para suportar a
geração automática das interfaces gráficas do SAD SustenAgro. Ela
permite associar tipos de dados às \foreignlanguage{english}{widgets}
(Web Components) necessárias para a geração automática das UI para
os SAD.
\item DSL que permite definir o comportamento de um SAD, usando os conceitos
das ontologias e integrando \foreignlanguage{english}{web components}
para gerar as Web UIs. Dentro dos elementos suportados estão a definição
dos objetos de avaliação, dos indicadores, das fórmulas do método
de avaliação e da definição do formato do relatório de cada análise.
\item Método e framework Decisioner para definir SADs baseadas em ontologias
e DSLs. O framework Decisioner possui uma arquitetura escalável que
permite generalizar a aplicabilidade dele a outros tipos de SADs com
características diferentes. Ele está baseado na linguagem \foreignlanguage{english}{Groovy}
e o framework \foreignlanguage{english}{Grails}. 
\item O SAD SustenAgro: sistema para avaliação da sustentabilidade em cana-de-açúcar,
que foi implementado usando o Framework Decisioner. O SAD SustenAgro
permite a reconfiguração, em tempo de execução, dos conceitos do domínio,
métodos de avaliação e componentes das interfaces gráficas, através
do \foreignlanguage{english}{Framework Decisioner}.
\item Resultados de avaliação do processo de design do SAD SustenAgro e
do seu protótipo final. Eles permitiram comprovar a realização dos
objetivos da pesquisa.
\item Artigo “\foreignlanguage{english}{Sustainability assessment of sugarcane
production systems: SustenAgro Method}” submetido ao periódico da
\foreignlanguage{english}{Elsevier “Energy for sustainable Development”}
ISSN: 0973-0826 submissão realizada em 23 de dezembro de 2016.
\item .
\end{enumerate}

\section{Dificuldades }

As dificuldades identificadas durante a realização da pesquisa, foram
as seguintes:

A falta de dados e informações para modelar o domínio do conhecimento
de avaliação em sustentabilidade, pelo que foi necessário realizar
o processo de modelagem para determinar que era necessário fornecer
ferramentas de definição do conhecimento por parte dos especialistas.
Porém, depois de estabelecer meios de representação de conhecimento
para os especialistas, encontrou-se que não existiam dados que representaram
instancias dos conceitos definidos, pelo que foi necessário criar
formulários web para fazer a recoleta de dados, dificultando o desenvolvimento
do projeto.

Outra dificuldade foi a falta de literatura especializada que forneça
um referencial em cada uma das etapas de desenvolvimento, repercutindo
na realização de tarefas erradas que consumiram tempo. Só foi possível
encontrar fundamentação teórica especializada em cada área do conhecimento
de maneira separada, no caso dos SAD as as fontes bibliográficas datam
de anos muito anteriores ao presente.

Além disso, o principal problema afrontado para a realização desta
pesquisa foi de carácter organizacional entre a equipe do projeto.
Este problema deve-se a que existem interesses heterogêneos por parte
da USP e da Embrapa, sendo o interesse em pesquisa por parte da USP
e de desenvolvimento por parte de Embrapa. Adicionalmente, não existia
uma definição especifica do tipo de sistema que os especialistas da
Embrapa precisavam, o que levou a realizar levantamento de requerimentos
e processos de design para especificar os sistemas requeridos, consumindo
tempo importante em processos que não estavam relacionados à pesquisa.

A Embrapa não permite que programas, desenvolvidos por ela ou em colaboração
com ela, sejam disponibilizado para o público em geral antes do registro
no INPI. Como o SAD SustenAgro foi uma colaboração entre Embrapa e
USP, a USP deve compartilhar esse registro com a Embrapa. Infelizmente
a burocracia da USP é muito demorada e esse registro está demorando
muito mais que o necessário. A Embrapa tem uma rede de produtores
de cana, usinas e cooperativas do setor interessados em sustentabilidade.
Apesar do SustenAgro ter sido avaliado por um número expressivo de
profissionais, seria muito interessante uma avaliação por usuários
finais de fazendas e usinas.

\section{Trabalhos futuros}

A partir do análise dos resultados e das dificuldades, foram identificadas
várias ideias que permitem complementar e melhorar os resultados da
pesquisa. Também a partir das seguintes ideias de trabalhos futuros
fica a possibilidade de definir projetos de pesquisa e desenvolvimento
tecnológico, tais como:

\subsection*{Metodologia de modelagem de conhecimento dos especialistas}

A metodologia e desenvolvimento das ontologias explicada na seção
\ref{sec:Ontologia-do-dom=0000EDnio}, permite propor uma metodologia
para acompanhar a modelagem de conhecimento de especialistas do domínio,
ela poderia estar composta das seguintes etapas:
\begin{enumerate}
\item Definição textual do conhecimento do domínio do especialista e identificação
dos elementos principais.
\item Modelar um mapa conceitual a partir dos elementos principais, que
permitam classificar os conceitos e estabelecer relacionamentos entre
eles.
\item Modelar a ontologia de maneira gráfica a través de um editor visual
de ontologias, que faça uso dos conceitos, classificações e relacionamentos
identificados nos mapas conceituais e permita definir regras em formato
\foreignlanguage{english}{OWL}
\item Gerar o formato computável da ontologia e carregar ela em um sistema
web para permitir o gerenciamento.
\item Por meio de serviços \foreignlanguage{english}{web} instanciar a ontologia
com indivíduos que representem dados reais do domínio para permitir
o uso dela e dos dados em outros sistemas.
\end{enumerate}
Dita metodologia facilitará a definição de conhecimento dos especialistas,
devido ao fato de fornecer uma abordagem gradual da modelagem.

\subsection*{Editor web de ontologias da web semântica.}

O Framework Decisioner integra um editor de ontologias baseado no
formato \foreignlanguage{english}{YAML}, o qual suporta a edição de
ontologias em tempo de execução, a través de uma web UI.

Este editor de ontologia pode ser complementado com um editor visual
de ontologias que permita a visualização e edição em forma de grafos,
um exemplo deste tipo de tecnologias é o visualizador de ontologias
WebVOWL \footnote{\selectlanguage{english}%
website do WebVOWL\foreignlanguage{brazil}{ \url{http://vowl.visualdataweb.org/webvowl.html}}\selectlanguage{brazil}%
}. 

Esta ferramenta integrará o método de modelagem de conhecimento proposto
anteriormente e terá um repositório online de conhecimento que permita
a reutilização.

\subsection*{Linguagem simplificado de consultas Sparql}

A \foreignlanguage{english}{DSL Interpreter} está composta de um módulo
que simplifica as consultas Sparql nas \foreignlanguage{english}{triplestores}.
Este módulo é um protótipo que pode ser melhorado e generalizado para
facilitar o armazenamento e recuperação de informações semânticas
por meio de uma linguagem simplificada em comparação do Sparql. 

Este componente pode ser disponibilizado como uma biblioteca que dê
suporte no desenvolvimento de software que façam uso do Sparql. 

\subsection*{Linguagem de edição de \foreignlanguage{english}{Web UIs}}

As \foreignlanguage{english}{UI} para os SAD, possuem vários tipos
de \foreignlanguage{english}{widgets}, que podem ser customizados
de acordo aos critérios dos especialistas. Para suportar a definição
da organização e apresentação das UI, é necessária uma linguagem que
gerencie as \foreignlanguage{english}{widgets} (elementos de uma \foreignlanguage{english}{UI}).
Esta linguagem será baseada em \foreignlanguage{english}{HTML} e permitira
usar termos simples do domínio, e gerará web UIs de tipo single-page
application.

Atualmente, foi definida uma DSL para especificar as web UI gerando
uma estrutura do documento em \foreignlanguage{english}{HTML}, que
associa \foreignlanguage{english}{templates} com o estilo predefinido
através de \foreignlanguage{english}{CSS} e com \foreignlanguage{english}{scripts}
definidos na linguagem \foreignlanguage{english}{Javascript}. Está
DSL está focada na estrutura do documento, mas a linguagem proposta
deve ser focada em termos dos domínio de interfaces gráficas em mais
alto nível, onde os elementos estejam modularizados. A estrutura hierárquica
da UI, estaria definida por meio de componentes relacionados semanticamente.

\subsection*{Linguagem genérico definição de SAD}

Decisioner DSL é uma linguagem de definição de SADs de avaliação,
que foi desenvolvido a partir do caso de uso SAD SustenAgro. Esta
DSL pode ser complementada e generalizada a partir da integração de
instruções e funcionalidades que façam parte de outros SAD diferentes
aos de avaliação, para desenvolver e validar uma linguagem geral que
dê suporte a definição de SADs.

\subsection*{Georreferenciamento}

Um aspecto importante para vários tipos de SAD são a associação de
termos e tecnologias de georreferenciamento, que são importantes para
relacionar o conhecimento de locais específicos e assim vincular dados
e informações para fazer sistemas com maior alcance.

No caso dos SADs associados a agricultura como SustenAgro, permitirá
contribuir na recuperação de dados existentes em fontes externas como
\foreignlanguage{english}{Geonames} \footnote{Site do GeoNames \url{http://www.geonames.org/}},
\foreignlanguage{english}{Wikimapia} \footnote{Site da Wikimapia \url{http://wikimapia.org/}}
e da \foreignlanguage{english}{DBpedia} \footnote{Site da \foreignlanguage{english}{DBpedia} \url{http://wiki.dbpedia.org/}}.
Os sistemas poderiam mapear as instâncias de conceitos agrícolas utilizando
coordenadas geográficas para \foreignlanguage{english}{GeoSPARQL}.

\subsection*{Biblioteca de \foreignlanguage{english}{web components}}

O Framework Decisioner conta com aproximadamente 60 \foreignlanguage{english}{web
components}, que dão suporte na geração de UI, as quais podem ser
agrupadas em uma biblioteca para permitir a reatualização delas e
a definição de web componentes especializados como os desenvolvidos
para o SAD SustenAgro.

A biblioteca \foreignlanguage{english}{Polymer} \footnote{Site do \foreignlanguage{english}{Polymer} \url{http://wiki.dbpedia.org/}}
suporta a criação de \foreignlanguage{english}{web components} reusáveis
para compor web UI. Dada as características dela recomenda-se implementar
a biblioteca com esta tecnologia. 

A partir desta biblioteca pode-se desenvolver um sistema de classificação
e busca de \foreignlanguage{english}{web components} para a geração
de SADs, inclusive mapeá-los diretamente desde a ontologia.

\subsection*{Biblioteca de \foreignlanguage{english}{Report} com suporte semântico}

Os SADs geram saídas em forma de relatórios, previsões e recomendações,
os quais devem ser flexíveis para facilitar a análise das informações
geradas, pelo qual se propõe o desenvolvimento de uma biblioteca especializada
em consultas complexas para a suportar as respostas dos SADs.
