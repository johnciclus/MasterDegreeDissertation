As principais contribuições desta pesquisa foram o método e ferramenta
de geração de SADs usando ontologias e \foreignlanguage{english}{DSLs},
Esse método permite complementar o conhecimento descrito em uma ontologia
com uma DSL, que define o comportamento e formato de um SAD. 

Ontologia e DSL estão abertas a edição pelos especialistas de domínio.
Eles podem editá-las, através de editores online (na própria ferramenta),
e modificar aspectos fundamentais do SAD em tempo real. Essa característica
permite um ciclo de desenvolvimento mais curto e uma participação
ativa dos especialistas de domínio. Como o Framework Decisioner fornece
uma aplicação semiacabada (como todo o framework) que os próprios
especialistas de domínio podem modificar, isso tende a diminuir os
custos associados a criação de SADs. 

O Framework Decisioner, o SAD SustenAgro e o método de definição de
SADs proposto tiveram uma avaliação positiva por parte dos especialistas
do domínio sustentabilidade e usuários finais da Embrapa e do projeto
SustenAgro. Essas avaliações demostraram que as ferramentas e o método
proposto solucionaram, para o caso do SustenAgro, os problemas identificados
na pesquisa e que os objetivos específicos desta pesquisa foram satisfeitos.

Deve-se ressaltar que, apesar dos bons resultados das avaliações e
do entusiasmo dos especialistas da Embrapa (que nunca tiveram tantas
facilidades para modificar o código de um projeto de SAD), ainda não
é possível, para especialistas de domínio, desenvolver um SAD, usando
o Decisioner, a partir do zero. Mas os resultados mostraram que eles
podem modificar muitos aspectos do sistema sem a ajuda de programadores.
O que é um avanço muito grande, se comparado com o que é possível
fazer em métodos de desenvolvimento tradicionais. Outra importante
contribuição foi a geração de uma versão funcional do SAD SustenAgro.
Ela permitiu cumprir os objetivos do projeto SustenAgro da Embrapa,
fornecendo uma ferramenta de avaliação da sustentabilidade em cana-de-açúcar
que pode efetivamente ser usada por empregados de fazendas e usinas.

Para continuar a pesquisa, foram sugeridos trabalhos futuros. Dentre
os mais relevantes estão a metodologia de criação de ontologias e
as ferramentas web para editá-las.


\section{Resultados}

Os resultados obtidos foram: 
\begin{enumerate}
\item Ontologia SustenAgro sobre avaliação de sustentabilidade em sistemas
produtivos de cana-de-açúcar na região centro-sul do Brasil. Ela representa
os principais conceitos desse domínio, necessários para o suporte
da avaliação de sustentabilidade. Dentre eles, estão os indicadores,
unidades produtivas, microrregiões, índices e métodos de avaliação.
A ontologia padroniza o conhecimento dos especialistas em um formato
computável. Ela foi desenvolvida em parceria com os especialistas
de domínio de sustentabilidade da Embrapa.
\item Ontologia sobre tipos de dados e controles visuais para suportar a
geração automática das interfaces gráficas dos SADs. Ela permite ass///////////ociar
tipos de dados às \foreignlanguage{english}{widgets} (Web Components)
necessárias para a geração automática das UI para os SAD.
\item DSL que permite definir o comportamento de um SAD, usando os conceitos
das ontologias, e integrar \foreignlanguage{english}{Web Components}
para gerar as Web UIs. Dentro dos elementos suportados, estão a definição
dos objetos de avaliação, de indicadores, das fórmulas do método de
avaliação e da definição do formato do relatório de cada análise.
\item Método e framework Decisioner para definir SADs, baseados em ontologias
e DSLs. O framework Decisioner possui uma arquitetura escalável para
permitir sua aplicabilidade a outros tipos de SADs (além do SustenAgro).
\item O SAD SustenAgro: sistema para avaliação da sustentabilidade em cana-de-açúcar,
que foi implementado usando o Framework Decisioner. O SAD SustenAgro
permite a reconfiguração, m tempo de execução, dos conceitos do domínio,
métodos de avaliação e componentes das interfaces gráficas, através
do \foreignlanguage{english}{Framework Decisioner}.
\item Resultados de avaliação do processo de design do SAD SustenAgro e
do seu protótipo final. Eles comprovaram que é possível gerar um SAD,
baseado no framework Decisioner, a partir de uma ontologia e DSL,
e ter sua Web UI gerada automaticamente. Eles também demonstram que
especialistas de domínio são capazes de alterar a ontologia e DSL,
sem a ajuda de programadores, e ver os resultados dessas modificações
em tempo real. Essa flexibilidade torna o desenvolvimento de SADs
mais agil com potencial de redução de seus custos finais.
\item Artigo “\foreignlanguage{english}{Sustainability assessment of sugarcane
production systems: SustenAgro Method}” submetido ao periódico da
\foreignlanguage{english}{Elsevier “Energy for sustainable Development”.}
ISSN: 0973-0826. Submissão realizada no dia 23 de dezembro de 2016.
\item Artigo ``SustenAgro Sistema de Apoio à Decisão baseado em Ontologias
e definido por uma Linguagem de Domínio Especifico'' submetido ao
periódico ``Revista Brasileira de Sistemas de Informação''. ISSN
Eletrônico: 1984-2902.
\end{enumerate}

\section{Dificuldades }

Durante a realização desta pesquisa foram identificadas as seguintes
dificuldades identificadas durante a realização da pesquisa, foram
as seguintes:

A primeira delas refere-se à falta de dados e informações para modelar
o domínio do conhecimento de avaliação em sustentabilidade. Depois
de estabelecer meios de representação para modelar o conhecimento
dos especialistas, descobriu-se que não existiam dados existentes
na Embrapa Meio Ambiente que representassem instâncias dos conceitos
definidos. Pelo que foi necessário fazer a recoleta de dados por parte
dos desenvolvedores, desenvolvendo formulários web e analisando as
dados enviados, este processo dificultou o desenvolvimento do projeto.

Outra dificuldade foi a falta de trabalhos de referência sobre a integração
de ontologias e DSLs,  o que levou a realizar alguns desenvolvimentos
de maneira errada que consumiram tempo da pesquisa. 

A dificuldade mais importante enfrentada foi de carácter organizacional,
entre a equipe do projeto do ICMC e da Embrapa. Esse problema deve-se
a uma diferença entre culturas de áreas diferentes, computação e sustentabilidade,
e da existência de interesses heterogêneos, por parte da USP e da
Embrapa. A Embrapa não desenvolve apenas pesquisas, mas usa essas
pesquisas em programas de extensão agrícola. Assim, aspectos como,
por exemplo, registro de software, que no mundo acadêmico tem importância
periférica, são muito mais importantes. A Embrapa não permite que
programas, desenvolvidos por ela ou em colaboração com ela, sejam
disponibilizados para o público em geral antes do registro no INPI
\footnote{Instituto Nacional da Propriedade Industrial \url{http://www.inpi.gov.br/}}.
Como o SAD SustenAgro foi uma colaboração entre Embrapa e USP, a USP
deve compartilhar esse registro com a Embrapa. Infelizmente, a burocracia
da USP é muito demorada e esse registro está demorando muito mais
que o necessário. A Embrapa tem uma comunidade de produtores de cana,
usinas e cooperativas do setor interessados em sustentabilidade. Apesar
do SustenAgro ter sido avaliado por um número expressivo de profissionais,
seria muito interessante uma avaliação por usuários finais de fazendas
e usinas.

\section{Trabalhos futuros}

A partir do análise dos resultados e das dificuldades, foram identificadas
várias ideias que permitirão complementar e melhorar os resultados
desta pesquisa. Também, a partir das seguintes ideias de trabalhos
futuros, fica a possibilidade de definir projetos de pesquisa e desenvolvimento
tecnológico.

\subsection*{Metodologia de modelagem de conhecimento dos especialistas}

A metodologia e desenvolvimento das ontologias, explicada na seção
\ref{sec:Ontologia-do-dom=0000EDnio}, permite propor uma metodologia
para acompanhar a um processo geral de modelagem de conhecimento de
especialistas do domínio. Ela seria uma nova proposta que poderia
estar composta das seguintes etapas:
\begin{enumerate}
\item Definição textual do conhecimento do domínio do especialista e identificação
dos elementos principais.
\item Modelagem de um mapa conceitual, a partir dos elementos principais,
que permita classificar os conceitos e estabelecer relacionamentos
entre eles.
\item Modelagem da ontologia de maneira gráfica, através de um editor visual
de ontologias, que faça uso dos conceitos, classificações e relacionamentos,
identificados nos mapas conceituais, e permita definir ontologias
em formato \foreignlanguage{english}{OWL}
\item Geração do formato computável da ontologia e carregá-la automaticamente
em um sistema web, para permitir o gerenciamento dela.
\item Instanciação da ontologia, por meio de serviços \foreignlanguage{english}{web,}
com instâncias que representem dados reais do domínio para permitir
o uso da ontologia e seus dados em outros sistemas.
\end{enumerate}
Essa metodologia facilitará a definição de conhecimento, por parte
dos especialistas, por fornecer uma abordagem mais gradual da modelagem.

\subsection*{Editor web de ontologias em OWL.}

O Framework Decisioner integra um editor de ontologias, baseado no
formato \foreignlanguage{english}{YAML}. Ele dá suporte a edição de
ontologias, em tempo de execução, através de uma Web UI. Ele pode
ser complementado com um editor visual de ontologias que permita a
visualização e edição em forma de grafos. Um exemplo desse tipo de
abordagem é o visualizador de ontologias WebVOWL \footnote{\url{http://vowl.visualdataweb.org/webvowl.html}}. 

O editor de ontologias em OWL integrará a metodologia de modelagem
de conhecimento, proposto anteriormente, e poderá ter um repositório
online de conhecimento que permita a reuso.

\subsection*{Linguagem simplificada de consultas Sparql}

O \foreignlanguage{english}{DSL Interpreter} tem um módulo que simplifica
as consultas Sparql nas \foreignlanguage{english}{triplestores}. Esse
módulo é um protótipo que pode ser melhorado e generalizado para facilitar
o armazenamento e recuperação de informações semânticas, por meio
de uma linguagem mais simples que Sparql. 

Esse módulo pode ser disponibilizado como uma biblioteca que dê suporte
ao desenvolvimento de programas que façam uso de Sparql. 

\subsection*{Linguagem de edição de \foreignlanguage{english}{Web UIs}}

As \foreignlanguage{english}{web UI} para os SADs possuem vários tipos
de \foreignlanguage{english}{web components}. É possível modificar
a aparência dessas web UIs, usando \foreignlanguage{english}{Cascading
Style Sheets (CSSs)} e uma DSL interna de uso dos programadores. Mas
esse tipo de modificação fica além do que especialistas podem fazer. 

Para suportar a definição e modificação da organização e apresentação
das UI, por parte dos especialistas, é necessária uma linguagem de
gerenciamento das \foreignlanguage{english}{widgets} (elementos de
uma \foreignlanguage{english}{UI}) com suporte a edição gráfica. 

\subsection*{Instanciação de outros SAD usando o Framework Decisioner}

O Framework Decisioner foi criado para permitir a instanciação de
uma classe de SADs que criam análises e recomendações para um cenário
específico. Mas ele foi desenvolvido a partir do caso de uso de apenas
um SAD desse tipo, o SustenAgro. Com a instanciação de outros SADs,
para áreas diferentes de sustentabilidade, a generalidade do framework
pode ser testada e, se necessário, mudanças implementadas. Instruções
e funcionalidades, que façam parte de outros SADs diferentes do SustenAgro,
podem ser integradas ao framework para generalizar seu escopo.

\subsection*{Georreferenciamento}

Um aspecto importante, para vários tipos de SADs, são a associação
de termos e tecnologias de georreferenciamento. Eles são importantes
para relacionar conhecimentos associados a locais específicos. No
caso dos SADs associados a agricultura, como SustenAgro, isso permitirá
a recuperação de dados existentes em fontes geográficas externas,
como \foreignlanguage{english}{Geonames} \footnote{ \url{http://www.geonames.org/}},
\foreignlanguage{english}{Wikimapia} \footnote{ \url{http://wikimapia.org/}}
e \foreignlanguage{english}{DBpedia} \footnote{ \url{http://wiki.dbpedia.org/}}. 

\subsection*{Biblioteca de \foreignlanguage{english}{Web Components}}

O Framework Decisioner conta com aproximadamente 60 \foreignlanguage{english}{Web
Components}. Eles dão suporte na geração de UIs, as quais podem ser
agrupadas em uma biblioteca, para permitir seu reuso e a definição
de Web Componentes especializados, como os desenvolvidos para o SAD
SustenAgro.

A biblioteca \foreignlanguage{english}{Polymer} \footnote{ \url{http://wiki.dbpedia.org/}}
suporta a criação de \foreignlanguage{english}{Web Components} reusáveis
para compor Web UIs. Dada as suas características recomenda-se a implementação
dessa biblioteca com essa tecnologia. 

\subsection*{}


\section{Agradecimento do financiamento}

Este trabalho foi realizado com uma bolsa de estudo financiado pela
Coordenação de Aperfeiçoamento de Pessoal de N\'{ı}vel Superior (CAPES)
no âmbito do Ministério da educação do Brasil.
