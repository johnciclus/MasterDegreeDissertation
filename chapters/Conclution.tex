Os resultados obtidos são: 
\begin{enumerate}
\item Ontologia em formatos da web semântica (RDF/OWL) da avaliação da sustentabilidade
no sistema de cana-de-açúcar. 
\item Protótipo de ontologia de interfaces gráficas em formatos da web semântica
(RDF/OWL) 
\item Linguagem de domínio especifico DSL, Decisioner, para definir e permitir
administrar os parâmetros e processos do sistema SustenAgro 
\item Protótipo de sistema de geração de interfaces suportado nas tecnologias
da web semântica 
\item Formulários para recolha de dados sobre sustentabilidade em cana-de-açúcar,
e o processo de colheita dos dados de algumas usinas do estado de
São Paulo 
\item Protótipo do sistema web Sustenagro que integra as ontologias e a
DSL, fornecendo um comportamento configurável em tempo de execução
dos parâmetros, processos e das interfaces gráficas de usuário
\end{enumerate}
Uma das finalidades deste projeto é construir um gerador de sistemas
de apoio a decisão que consiga suportar outros domínios de conhecimento,
propondo para a comunidade uma metodologia de desenvolvimento e manutenção
que fique simples para os usuários finais e assim permitir que os
especialistas no domínio façam as mudanças sem precisas dos especialistas
de T.I.

\section{Discussão}

\section{Trabalhos Futuros}

Analisando os componentes e o desenvolvimento do Sistema SustenAgro,
surgem ideias que permitem expandir o trabalho atual e propor projetos
de investigação e desenvolvimento tecnológico, entre as possibilidades
identificadas estão:

\subsection{Editor de ontologias web focado no usuário especialista}

No sistema SustenAgro foi integrado um editor de ontologias que é
baseado no formato YAML de fácil legibilidade, com a finalidade de
fornecer uma ferramenta web de fácil uso e acesso para que os usuários
especialistas possam editar a ontologia do domínio, no caso de SustenAgro
a ontologia de avaliação de sustentabilidade.

A partir de esta ferramenta é possível propor um editor online focado
em especialistas do domínio, com a finalidade de facilitar a definição
e edição de ontologias, o qual poderia estar composto pelas seguintes
funcionalidades:
\begin{enumerate}
\item Editor textual do conhecimento do domínio do especialista
\item Editor de mapas conceituais que reuse informações do texto
\item Editor visual da ontologia que reuse os mapas conceituais e permita
gerar o modelo hierárquico RDFS
\item Editor de regras e OWL
\item Editor de indivíduos
\item Publicação da ontologia em um repositório permanente
\end{enumerate}

\subsection{Linguagem de definição de SAD: Decisioner}

No sistema SustenAgro, foi desenvolvida uma DSL que permite definir
o comportamento de um SAD, no caso realizado foi o desenvolvimento
de um sistema de avaliação de sustentabilide, os conceitos foram abstraidos
por meio da ontologia de SAD que permite identificar e modelar os
conceitos gerais dos sistemas software de avaliação usados pela Embrapa
Meio Ambiente, o que resultou em uma linguagem de proposito especifico
para o sistema SustenAgro e similares.

Esta linguagem pode ser validada, melhorada e expandida por meio da
definição e desenvolmiento de outros SAD com propositos similares,
desenvolvimento continuado pelo sistema NanoTec Ram que permite a
valiação do risco.

\subsection{Reportes de consultas semanticas}

Gerar SADs que suportem consultas complexas que requerem conhecimento
semântico, facilitando o processo de análises da informação por parte
dos usuários.

\subsection{Linguagem de edição de VIEWs}

Na geração de interfaces gráficas é necessário o gerenciamento e organização
do conteúdo e das widgets (elementos de uma GUI) com a finalidade
de fornecer uma representação visual que consiga informar um determinado
assunto.

Atualmente na geração de VIEWs nos sistemas web, são usadas tecnologias
feitas para os desenvolvedores web front-end que permitem a manipulação
da estrutura do documento atraves de HTML, do estilo do documento
atraves de CSS e do comportamento JS. Mas estes elementos estão focados
na representação computacional dessas carateristicas, porem é possivel
propor outra maneira de representação em mais alto nivel.

A proposta é de uma linguagem para ediação de VIEWs, onde os elementos
estejam modularizados permitindo abstrair a View em uma hierarquia
de componentes relacionando-os semanticamente por meio de uma linguagem
de alto nivel de usuário não especialista em web development.

\subsection{Organização de widgets por meio de Polymer}

No desenvolvimento do Sistema SustenAgro, foram definidas 60 widgets
que foram reusadas na definição de cada uma das vistas de SustenAgro,
essas widgets estão definidas no formato de templates do framework
Grails

\subsection{Georeferenção nos SAD}
\begin{enumerate}
\item Recuperação dos dados dos repositórios FAO Linked Data 
\item Utilização de dados de fontes externas como Geonames, Wikimapia, DBpedia
\item Mapeamento dos dados para uma \textit{triplestore} utilizando coordenadas
geográficas para GeoSPARQL 
\end{enumerate}

\subsection{Desenvolver um framework para geração das interfaces usuário-computador
geral para SAD. }

\subsection{Fornecer um sistema de classificação e busca de controles de interfaces
usuário-computador}

\section{Dificuldades e Limitações}

Até o presente momento, foi evidenciado como dificuldade para desenvolvimento
do projeto as escassas fontes de informação que forneçam uma conexão
entre sistemas de produção agrícola e sustentabilidade. Só foi possível
encontrar fontes de informação especializada em cada área do conhecimento
de maneira separada. Outro problema é a falta de dados resultantes
da aplicação dos indicadores de sustentabilidade fornecidos pela Embrapa.

Organizacional pela equipe

Interesse heterogêneos por parte da USP e da Embrapa

Cada uma das anteriores etapas de desenvolvimento foram testadas com
os especialistas do domínio, para conseguir realizar o correto levantamento
de requisitos, devido a que não existia uma definição especifica do
que os especialistas precisavam.
