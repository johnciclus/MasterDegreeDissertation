Os Sistemas de Apoio à Decisão \nomenclature{SAD}{Sistemas de Apoio à Decisão}
processam e organizam os dados e informações para gerar resultados
de valor que suportam o processo de decisão, estes sistemas tem evoluído
junto com a Web \citet{Shim2002}, integrando novas técnicas no processamento
dos dados, tecnologias na representação visual de resultados e no
uso colaborativo por parte dos usuários; os SADs são desenvolvidos
para um domínio de conhecimento específico e para usuários especialistas,
no caso da pesquisa apresentada neste documento, foi desenvolvido
um SAD intitulado SustenAgro baseado na web semântica e cujo domínio
de conhecimento foi a Avaliação da Sustentabilidade em Agricultura.

Os sistemas produtivos agrícolas são sistemas complexos que integram
fenômenos de natureza diversa \citep{simon1991architecture}, dentro
deles está o sistema produtivo de cana-de-açúcar que é extremamente
importante para a economia do Brasil e do estado de São Paulo, devido
a que é uma das principais culturas produzidas no pais\citet{Storquato2015},
pelo qual foi escolhido como sistema produtivo piloto para abordar
a presente pesquisa.

Dada a complexidade dos sistemas de produção agrícola, surgiu a necessidade
de determinar um método de avaliação da sustentabilidade de maneira
integral\citet{Singh2012281}, por esta razão pesquisadores da Embrapa
Meio Ambiente desenvolveram um método de avaliação de sustentabilidade
que aborda a avaliação em termos de indicadores, medindo aspectos
críticos no sistema produtivo, para determinar se é ou não é sustentável,
e assim gerar recomendações para tomar medidas corretivas.

A partir do método de avaliação, foi desenvolvida uma ferramenta de
avaliação da sustentabilidade intitulada software SustenAgro que implementa
o método SustenAgro por meio de um sistema de apoio a decisão e que
consegue adaptar-se às mudanças do domínio, o qual está baseado em
uma ontologia que permitem representar e estruturar o conhecimento
de avaliação da sustentabilidade em agricultura, gerando uma representação
mais exata da realidade do que outros modelos de dados, em formato
de ontologias da web semântica, sobre o qual é possível fazer inferências
e assim gerar informações para suportar a decisão.

Também foi desenvolvida a ontologia de representação das interfaces
gráficas, permitindo um mapeamento entre conceitos do domínio dos
especialistas e as interfaces gráficas, concedendo flexibilidade ao
sistema para adaptar-se às mudanças dos conceitos do domínio de conhecimento.

As ontologias forneceram uma representação de conhecimento, que foi
gerenciada desde uma \nomenclature{DSL}{Domain Specific Language},
desenhada para gerar sistemas de apoio a decisão, dita linguagem e
o interprete dela foram intitulados Decisioner, os quais constituem
uma ferramenta geradora de sistemas de avaliação.

O desenvolvimento do projeto foi apoiado pela Embrapa\nomenclature{Embrapa}{Empresa Brasileira de Pesquisa Agropecuária}
Meio Ambiente, instituição parceira que planejou e executou o projeto
SustenAgro, cuja finalidade é fornecer os fundamentos teóricos para
suportar avaliações que possibilitem o embasamento de politicas públicas
que incentivem a sustentabilidade no setor produtivo da cultura de
cana-de-açúcar no centro-sul do Brasil.

\section{Motivação}

O conhecimento sobre avaliação da sustentabilidade da cultura de cana-de-açúcar
está em continua mudança \citep{oliveira:2013}, porém foi necessária
uma representação desse conhecimento que adapte-se às mudanças e que
facilite a comunicação entre os especialistas do domínio e os desenvolvedores
de software, neste sentido as ontologias da web semântica satisfazem
este requisito porque separam o conhecimento do domínio da lógica
da computação, permitindo abordar cada desenvolvimento de uma maneira
independente e suportando conceitos importantes como a inferência,
que são de grande importância no desenvolvimento como os \nomenclature{SAD}{Domain Specific Language}.

Uma caraterística importante de SustenAgro é a recuperação da informação
com significado semântico, permitindo que o sistema dê respostas às
consultas complexas de interesse para os especialistas, além da integração
com conhecimento externo existente em formatos da web semântica, como
são os padrões\nomenclature{RDF}{Resource Description Framework},
\nomenclature{OWL}{Web Ontology Language} e vários sistemas de representação
do conhecimento como dicionários, thesaurus e redes semânticas, o
que permite aumentar as possibilidades de integração com diversas
tecnologias e fornecer novas funcionalidades. 

O processo de modelagem da ontologia requiriu uma aprendizagem especifica
guiada por especialistas e depois foi necessária a implementação de
uma\nomenclature{DSL}{Domain Specific Language} a qual permite aos
administradores da ferramenta definir como são usados e apresentados
os conceitos da ontologia por meio de elementos da interface gráfica
do usuário, para assim fornecer um sistema adaptável às mudanças do
domínio, em termos de interface gráfica durante runtime.

A DSL e o interprete conformam uma ferramenta intitulada Decisioner,
que por meio de umas declarações permite a geração de \nomenclature{SAD}{Sistema de apoio à decisão},
dita ferramenta foi a principal contribuição desta pesquisa.

\section{Objetivo}

Desenvolver um sistema web que permita realizar a avaliação da sustentabilidade
nos sistemas produtivos de cana-de-açúcar no centro-sul do Brasil
fazendo uso das tecnologias da web semântica para representar o conhecimento
dos especialistas e flexibilizar a geração da interface gráfica de
usuário.

Para atingir o objetivo proposto, foi necessário atingir os seguintes
objetivos específicos: 

\subsection*{Objetivos específicos}
\begin{itemize}
\item Desenvolver uma ontologia sobre avaliação da sustentabilidade nos
sistemas produtivos de cana-de-açúcar do centro sul do Brasil, como
base conceitual e tecnológica do sistema SustenAgro.
\item Desenvolver uma ontologia sobre controles visuais de interfaces gráficas
para suportar a definição dos indicadores do sistema SustenAgro.
\item Desenvolver uma DSL que gerencie as ontologias e que flexibilize a
definição da interface de usuário por parte dos administradores do
sistema.
\item Demonstrar que um sistema de apoio na tomada de decisões baseado nas
tecnologias da web semântica, permite a realização de consultas complexas
que requerem conhecimento semântico, facilitando o processo de análises
da informação por parte dos usuários.
\item Definir uma arquitetura para sistemas de avaliação baseados em conhecimento
de domínios específicos.
\end{itemize}

\section{Resultados Principais}

Os principais resultados desta pesquisa e projeto de mestrado são:
\begin{itemize}
\item Ontologia sobre avaliação de sustentabilidade em cana-de-açúcar, representando
os principais conceitos desse domínio: indicadores, os índices e o
método de avaliação; permitindo assim suportar o desenvolvimento das
outras tecnologias do presente projeto.
\item Ontologia sobre interfaces gráficas que permite representar os tipos
de dados e \foreignlanguage{english}{widgets} necessários para a geração
dos Sistemas de Apoio na Decisão.
\item DSL: linguagem de domínio especifico que permite gerenciar ontologias
para definir sistemas de apoio à decisão.
\item Decisioner: Sistema gerador de SADs, que suporta a integração de ontologias,
DSL e tecnologias da web semântica.
\item SustenAgro: Sistema de Apoio a Decisão para avaliar a sustentabilidade
em cana-de-açúcar, implementado o método SustenAgro e tecnologias
da web semântica.
\end{itemize}

\section{Organização}

Este trabalho de dissertação está estruturado da seguinte forma: 

Capítulo 2: Apresenta a fundamentação teórica sobre a Web Semântica
com a finalidade de descrever as principais tecnologias e a teoria
necessária para desenvolver o presente trabalho.

Capítulo 3: Apresenta a fundamentação sobre DSL e como pode ser usada
em associação com a Web Semântica para gerar novos tipos de sistemas
de apoio à decisão.

Capítulo 4: Apresenta os trabalhos relacionados sobre geração automática
de Sistemas de Apoio à Decisão e os Sistemas de avaliação da Sustentabilidade
para representar o estado da arte da presente pesquisa. 

Capítulo 5: Apresenta a abordagem da metodologia e o desenvolvimento
de cada um dos produtos software resultantes do presente trabalho. 

Capítulo 6: Apresenta a avaliação realizada pelos especialistas.

Capítulo 7: Apresenta os resultados obtidos

Capítulo 8: Apresenta as conclusões do presente trabalho, uma discussão
e possíveis trabalhos futuros.

Finalmente são apresentados os anexos que descrevem componentes específicos
do trabalho.
