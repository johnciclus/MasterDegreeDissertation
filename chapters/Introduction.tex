Os Sistemas de Apoio à Decisão (SAD ) organizam e processam os dados
e informações para gerar resultados de valor que apoiem e melhorem
a tomada de decisão em um área de conhecimento, também denominada
como domínio especifico \citep{Turban:2004:DSS:994103}. Os SADs integram
conhecimento desenvolvido pelos especialistas do domínio que fica
implícito nos dados, informações e processos usados \citep{power2000web}.
Tal conhecimento, específico de um domínio, não é familiar para desenvolvedores
de software. Eles têm que usar técnicas diversas para o levantamento
de requisitos para entender o domínio dos especialistas e assim implementar
o sistema software corretamente \citep{GavrilovaAndreeva2012}. Quanto
mais especializado for o domínio dos especialistas, mais esforço adicional
será necessário aos desenvolvedores do sistema, o que leva a ampliação
do tempo e custo de desenvolvimento. 

Adicionalmente, os especialistas do domínio, em geral, não tem conhecimento
suficiente em matéria de desenvolvimento de sistemas de software para
realizar o dito processo por eles mesmos. Além disso, os dois domínios,
tanto dos especialistas de domínio como dos desenvolvedores de software,
são tão amplos que precisam de perfis particulares para realizar os
processos corretamente \citep{roussey2010ontologies}. Dentro deste
contexto, foi identificado o problema da inexistência de um meio de
representação de conhecimento para definir SADs, que tenha um formato
computável, entendível e acessível pelos especialistas do domínio.

Como exemplo do problema anterior, podemos expor o caso dos especialistas
em sustentabilidade da Embrapa Meio Ambiente, que desenvolveram o
projeto SustenAgro (capítulo \ref{chap:Sustainability_Assessment}).
Nesse projeto, foi desenvolvido um método de avaliação de sustentabilidade
no sistema produtivo da cana-de-açúcar do centro-sul do Brasil \citep{oliveira:2013}.
Tais especialistas precisavam implementar um SAD para disponibilizar
o método SustenAgro à comunidade interessada em realizar avaliações
de sustentabilidade em cana-de-açúcar. Nesse caso, foi identificado
que eles possuíam o conhecimento do domínio de avaliação de sustentabilidade
da cana-de-açúcar, mas não tinham um meio para definir esse conhecimento
de maneira computável em um SAD.

\section{Motivação}

A pesquisa em representação e organização de conhecimento tem alto
impacto devido ao fato de fornecer métodos e ferramentas para entender
e gerenciar o conhecimento em diversos domínios \citep{Tudhope2006}.
Especificamente nos SADs, ela pode fornecer meios de integração de
conhecimento que aumentam as funcionalidades e a eficiência desses
sistemas, inclusive trazendo vantagens no processo de desenvolvimento
dos SAD \citep{Saxena}.

A análise do projeto SustenAgro permitiu identificar as seguintes
motivações relacionadas com a integração de conhecimento dos especialistas
de domínio:

A primeira delas é a necessidade de um método e ferramenta para que
os especialistas do domínio definam o conhecimento nos SADs, principalmente
as características particulares que requerem profundo conhecimento.
Isso permite a participação deles como descritores de conhecimento
especifico fornecendo, aos desenvolvedores de software, tempo adicional
para dedicar-se aos assuntos próprios da computação e assim agilizar
o processo de desenvolvimento de SADs. Essa abordagem facilita a definição
de SADs com menos intervenção por parte dos não especialistas do domínio,
fazendo que a definição do conhecimento fique em termos conhecidos
pelos especialistas e seja gerenciada por eles mesmos. 

A segunda refere-se a fornecer meios computáveis de representação
desse conhecimento, que facilitem a comunicação entre os especialistas
do domínio e os desenvolvedores de software e que adaptem-se às mudanças
do domínio. Domínios de conhecimento estão em continua mudança, como
é o caso do domínio avaliação da sustentabilidade da cultura de cana-de-açúcar\citep{oliveira:2013}. 

A terceira é o impacto que gera o correto uso dos SAD\citep{Lee2008349}.
O desenvolvimento do SAD SustenAgro, em cooperação com a Embrapa Meio
Ambiente, poderia gerar impacto positivo na produção agrícola, já
que ele fornece a funcionalidade de avaliar a sustentabilidade na
cana-de-açúcar. Este SAD em particular é uma contribuição inédita
que também permite a aplicação em outros sistemas agrícolas, o que
poderia suportar melhoras e correções na produção de alimentos e produtos
neste setor \citep{Matthews2008149}.

Para definir um meio computável de representação de conhecimento,
foram analisados vários tipos de sistemas de organização de conhecimento
ou \foreignlanguage{english}{Knowledge Organization System} (\foreignlanguage{english}{KOS}\nomenclature{KOS}{Knowledge Organization System}).
Um dos \foreignlanguage{english}{KOS} mais completos são as ontologias
porque permitem definir, classificar, relacionar e inferir conhecimento.
Por este motivo as ontologias foram selecionadas para representar
a modelagem do domínio dos especialistas.

Numa parceria com a Embrapa Meio Ambiente, foi desenvolvida uma ontologia
de avaliação da sustentabilidade em cana-de-açúcar junto com os especialistas
desta instituição. Nas áreas de biologia e agricultura as ontologias
são comuns, desta forma eles não tiveram dificuldade com o conceito.
Eles usaram uma \foreignlanguage{english}{Domain Specific Language}
(DSL) \citep{fowler2010domain} que permite formatar as perguntas
aos usuários finais, definir o método de avaliação e gerar o relatório
de cada análise. Os especialistas apreciaram muito a utilização de
uma ontologia em um SAD, foi a primeira vez que fizeram isso, e particularmente,
o fato de que as mudanças na ontologia, façam mudanças imediatamente
nos componentes do SAD.

Além das motivações vinculadas com o domínio dos especialistas em
sustentabilidade existe a motivação de desenvolver novas tecnologias.
Uma contribuição tecnologia neste sentido na pesquisa foi o Framework
Decisioner composto por ontologias e \foreignlanguage{english}{DSLs},
cujo proposito é facilitar a implementação de SADs, sendo este framework
instanciado para produzir o SAD SustenAgro.

\section{Objetivo}

Desenvolver um método e ferramenta web, baseados em ontologias, que
permitam representar o conhecimento de especialistas do domínio para
suportar a definição de SADs. 

\subsection*{Objetivos específicos}

Para alcançar o objetivo proposto, foi necessário cumprir os seguintes
objetivos específicos: 
\begin{itemize}
\item Desenvolver uma ontologia sobre avaliação da sustentabilidade nos
sistemas produtivos de cana-de-açúcar do centro-sul do Brasil.
\item Desenvolver uma DSL que permita a definição da interface de usuário,
das formulas do modelo, usado pelos especialistas, e do formato do
relatório final de cada análise.
\item Definir a arquitetura e o código da ferramenta Decisioner para gerar
SADs baseados em conhecimento de domínios específicos. Os SADs gerados
usam uma ontologia, desenvolvida por especialistas do domínio e uma
DSL, para gerar as interfaces e funcionalidades do sistema.
\item Demonstrar que o método e ferramenta Decisioner permitem a criação
de SADs funcionais que podem ser modificados por especialistas de
domínio. Esses especialistas devem definir a ontologia e modificar
o comportamento do sistema com pouca ou nenhuma intervenção de desenvolvedores.
\end{itemize}

\section{Resultados Principais}

As principais contribuições da pesquisa foram:
\begin{itemize}
\item Ontologia SustenAgro sobre avaliação de sustentabilidade em cana-de-açúcar,
representando os principais conceitos desse domínio: indicadores,
índices e métodos de avaliação. Ela foi desenvolvida em parceria com
os especialistas de domínio de sustentabilidade da Embrapa.
\item Ontologia sobre controles visuais para suportar a geração automática
das interfaces gráficas do SAD SustenAgro. Ela permite associar tipos
de dados aos \foreignlanguage{english}{web components} que constituem
as interfaces dos SADs.
\item Uma DSL baseada na linguagem \foreignlanguage{english}{Groovy} que
permite a definição da interface de usuário, do formato do dados a
processar, das fórmulas do modelo e do formato do relatório final
de cada análise.
\item Método e framework Decisioner para definir SADs por parte dos especialistas
do domínio, desenvolvido a través do framework \foreignlanguage{english}{Grails}.
\item O SAD SustenAgro: Sistema para avaliação da sustentabilidade em cana-de-açúcar,
na região centro-sul do Brasil, instanciada a través do Framework
Decisioner.
\item Artigo ``\foreignlanguage{english}{Sustainability assessment of sugarcane
production systems: SustenAgro Method}'' submetido ao periódico da
\foreignlanguage{english}{Elsevier} ``\foreignlanguage{english}{Energy
for sustainable Development}'' ISSN: 0973-0826.
\item Artigo ``SustenAgro Sistema de Apoio à Decisão baseado em Ontologias
e definido por uma Linguagem de Domínio Especifico'' submetido ao
periódico ``Revista Brasileira de Sistemas de Informação'' ISSN
Eletrônico: 1984-2902.
\item Resultados positivos das avaliações do Framework Decisioner, do método
de geração de SADs e do Sistema SustenAgro.
\end{itemize}

\section{Organização da dissertação.}

A presente dissertação está estruturada da seguinte forma:

Capítulo \ref{chap:SAD}: Apresenta a definição de Sistemas de Apoio
à Decisão, as características do SAD SustenAgro e os trabalhos relacionados,
basicamente um estado da arte da presente pesquisa.

Capítulo \ref{chap:SemanticWebAndDSLs}: Apresenta as Ontologias da
web semântica e \foreignlanguage{english}{DSLs}, com a finalidade
de descrever as principais tecnologias e a teoria necessária para
desenvolver a presente pesquisa.

Capítulo \ref{chap:Decisioner}: Apresenta o protótipo do Framework
Decisioner que suporta a geração de Sistemas de Apoio à Decisão. 

Capítulo \ref{chap:SustenAgro}: Apresenta o SAD SustenAgro, desenvolvido
na presente pesquisa como caso de uso do Framework Decisioner, permitindo
desenvolver a arquitetura e demostrar a funcionalidade do Framework
desenvolvido.

Capítulo \ref{chap:Avalia=0000E7=0000E3o}: Apresenta a avaliação
realizada pelos especialistas em cada um dos sistemas software desenvolvidos
e no método de geração de SAD.

Capítulo \ref{chap:Conclus=0000E3o}: Apresenta as conclusões do presente
trabalho, dificuldades, trabalhos futuros e uma discussão sobre a
pesquisa em geral.

Finalmente são apresentados os anexos que descrevem conceitos de terceiros
usados no trabalho e informações técnicas dos sistemas desenvolvidos.
