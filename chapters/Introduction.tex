Os Sistemas de Apoio à Decisão (SAD \nomenclature{SAD}{Sistemas de Apoio à Decisão})
organizam e processam os dados e informações para gerar resultados
de valor que auxiliem o processo de decisão em um domínio especifico,
ditos sistemas integram conhecimento desenvolvido pelos especialistas
do domínio que fica implícito nos dados, informações e processos,
dito conhecimento não é familiar para os desenvolvedores de software,
o que leva a usar diversas técnicas de levantamento de requerimentos
que envolvem o aprendizado de tópicos do domínio dos especialistas
por parte dos desenvolvedores, este processo exige um esforço adicional
por parte dos desenvolvedores do sistema e traz limitações no tempo
e custo de desenvolvimento, devido a que o conhecimento precisa ser
explicado por parte dos especialistas do domínio aos desenvolvedores
de software, para que eles consigam entender o conhecimento e assim
implementar o sistema software corretamente.

Adicionalmente, os especialistas do domínio pelo geral não tem o conhecimento
de desenvolvimento de sistemas software para realizar dito processo
por eles mesmos; alem disso os dois domínios, tanto dos especialistas
do domínio como do desenvolvimento de software são tão amplos que
precisam perfis particulares para realizar os processos corretamente,
devido ao anterior foi identificado o problema da inexistência de
uma representação de conhecimento para definir SADs, que tenha um
formato computável, entendível e acessível pelos especialistas do
domínio e desenvolvedores de software.

Como exemplo do anterior problema, podermos expor o caso dos especialistas
em sustentabilidade da Embrapa Meio Ambiente, que desenvolveram o
projeto SustenAgro (capitulo \ref{chap:Sustainability_Assessment}),
onde foi desenvolvido um método de avaliação de sustentabilidade no
sistema produtivo de cana-de-açúcar, os especialistas precisavam implementar
um SAD para disponibilizar o uso do método à comunidade interessada
em realizar avaliações de sustentabilidade em cana-de-açúcar, , no
caso deles foi identificado que tinham o conhecimento do domínio avaliação
de sustentabilidade da cana-de-açúcar, mas não tinham o método e ferramenta
software para definir dito conhecimento de maneira computável em um
SAD, pelo que dito problema e caso de uso foram abordados na presente
pesquisa como projeto piloto.

\section{Motivação}

A pesquisa em representação e organização de conhecimento tem alto
impacto devido a fornece métodos e ferramentas para gerenciar o conhecimento
em diversos domínios, especificamente no desenvolvimentos dos SADs,
pode fornecer meios de integração de conhecimento que aumentam as
funcionalidades e a eficiência desses sistemas em comparação com os
métodos tradicionais de desenvolvimento.

Sobre o caso especifico do projeto SustenAgro, existem varias motivações
para desenvolver novos meios de definição de SAD, entre eles está
principalmente o fornecimento de um método e ferramenta para que os
especialistas do domínio definiam o conhecimento nos SADs, permitindo
a participação deles como descritores de conhecimento especifico,
também temos que a avaliação da sustentabilidade da cultura de cana-de-açúcar
está em continua mudança \citep{oliveira:2013}, pelo que existe a
necessidade de fornecer meios computáveis de representação desse conhecimento
que adaptem-se às mudanças do domínio e que facilite a comunicação
entre os especialistas do domínio e os desenvolvedores de software.

Além disso, permitira definir SADs com menos intervenção por parte
dos não especialistas do domínio, fazendo que a definição do conhecimento
fique em termos dos especialistas e gerenciadas por eles mesmos, fornecendo
a possibilidade de que descrevam características particulares que
requerem profundo conhecimento do domínio, finalmente fornecera aos
desenvolvedores tempo adicional para dedicar-se aos assuntos próprios
da computação, e assim agilizar o processo de desenvolvimento de SADs.

Na representação de conhecimento existem vários tipos de sistemas
de organização de conhecimento (\nomenclature{KOS}{Knowledge Organization System}
pelas siglas em inglês), um dos sistemas mais completos são as ontologias
que permitem definir, classificar, relacionar e inferir conhecimento,
e como deseja-se um meio que suporte vários aspectos do conhecimento,
se definiu usar este tipo de \foreignlanguage{english}{KOS} no processo
de modelagem, fornecendo um caso real de aprendizagem e implementação
de ontologias.

A Embrapa Meio Ambiente também tem outros SAD que podem ser avaliados
e modelados com a finalidade de desenvolver um método e ferramenta
geral de definição de SADs.

\section{Objetivo}

Desenvolver um método e ferramenta web baseados em ontologias que
permita representar o conhecimento dos especialistas do domínio para
suportar definição de SADs, e provar o funcionamento por meio da definição
do SAD SustenAgro, que tem como finalidade suportar a avaliação da
sustentabilidade nos sistemas produtivos de cana-de-açúcar no centro-sul
do Brasil 

Para atingir o objetivo proposto, foi necessário atingir os seguintes
objetivos específicos: 

\subsection*{Objetivos específicos}
\begin{itemize}
\item Desenvolver um método de definição de SAD por parte dos especialistas.
\item Definir uma arquitetura e ferramenta para definir SADs baseados em
conhecimento de domínios específicos.
\item Desenvolver uma ontologia sobre avaliação da sustentabilidade nos
sistemas produtivos de cana-de-açúcar do centro sul do Brasil, como
base conceitual e tecnológica do sistema SustenAgro.
\item Desenvolver uma ontologia sobre controles visuais para suportar a
geração da interface gráfica do SAD SustenAgro.
\item Desenvolver uma DSL que gerencie as ontologias e que flexibilize a
definição da interface de usuário por parte dos administradores do
sistema.
\item Demonstrar que o método e ferramenta de definição de SADs por parte
dos especialistas, permite a geração de sistemas funcionais.
\end{itemize}

\section{Resultados Principais}

Os principais resultados desta pesquisa e projeto de mestrado são:
\begin{itemize}
\item Método e ferramenta para definir SADs por parte dos especialistas
do domínio.
\item Ontologia sobre avaliação de sustentabilidade em cana-de-açúcar, representando
os principais conceitos desse domínio: indicadores, os índices e o
método de avaliação; permitindo assim suportar o desenvolvimento das
outras tecnologias do presente projeto.
\item Ontologia sobre interfaces gráficas que permite representar os tipos
de dados e \foreignlanguage{english}{widgets} necessários para a geração
dos Sistemas de Apoio na Decisão.
\item DSL: linguagem de domínio especifico que permite gerenciar ontologias
para definir sistemas de apoio à decisão.
\item Protótipo do Decisioner: Sistema gerador de SADs, que suporta a integração
de ontologias e DSL em ambientes web.
\item SustenAgro: Sistema de Apoio a Decisão para avaliar a sustentabilidade
em cana-de-açúcar, implementado o método SustenAgro e tecnologias
da web semântica.
\item Artigo ``\foreignlanguage{english}{Sustainability assessment of sugarcane
production systems: SustenAgro Method}'' submetido no periódico acadêmico
``\foreignlanguage{english}{Energy for sustainable Development}''
ISSN: 0973-0826 submetido na data 23 de dezembro do 2016. 
\end{itemize}

\section{Organização}

Este trabalho de dissertação está estruturado da seguinte forma: 

Capítulo 2: Apresenta o SAD SustenAgro e os trabalhos relacionados
sobre geração de Sistemas de Apoio à Decisão e os Sistemas de Avaliação
da Sustentabilidade para representar o estado da arte da presente
pesquisa.

Capítulo 3: Apresenta a fundamentação teórica sobre Ontologias e DSL
com a finalidade de descrever as principais tecnologias e a teoria
necessária para desenvolver o presente trabalho.

Capítulo 4: Apresenta o protótipo do sistema Decisioner que permite
suportar a geração de Sistemas de Apoio à Decisão. 

Capítulo 5: Apresenta o SAD SustenAgro, desenvolvido na presente pesquisa
e que se serviu como primeiro caso de uso do sistema Decisioner, para
definir a arquitetura dele e demostra a funcionalidade do sistema
desenvolvido.

Capítulo 6: Apresenta a avaliação realizada pelos especialistas.

Capítulo 7: Apresenta as conclusões do presente trabalho, uma discussão
e possíveis trabalhos futuros.

Finalmente são apresentados os anexos que descrevem conceitos específicos
do trabalho.
