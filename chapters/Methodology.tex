Este capítulo apresenta a metodologia realizada no desenvolvimento
do presente projeto entre eles a ontologia de domínio do SustenAgro
e artefatos para o desenvolvimento da interface visual do sistema:
User Stories, Scenarios, Story Boards, Mockups e um protótipo para
a interface do SustenAgro.

\section{Ontologia de Domínio do SustenAgro}

Eles são sistemas complexos que integram fenômenos de natureza diversa
\citep{simon1991architecture}, integrando três subsistemas: (i) o
subsistema ambiental que fornece as condições físicas, químicas e
biológicas que suportam o desenvolvimento das culturas, (ii) o subsistema
social que integra organizações e pessoas que realizam a produção,
relacionando-se internamente e externamente com os sistemas produtivos
e (iii) o subsistema econômico que estabelece as condiciones de oferta
e demanda dos produtos e subprodutos do sistema de produção agrícola;das
interações entre estes subsistemas, emerge um comportamento complexo
que requer uma abordagem holística e inter-relacionada para suportar
a tomada de decisões que garantam a sustentabilidade do sistema em
analise.

ditos sistemas também são chamados dimensões da sustentabilidade,
segundo a literatura estas dimensões são: ambiental, econômica e social
\citep{AlkanOlsson:2009}.

O software SustenAgro baseou-se em indicadores da sustentabilidade
nas tres dimensões, os quais foram propostos por um grupo de especialistas
de diversas áreas da produção agrícola e sustentabilidade \citep{oliveira:2013},
esta base conceitual foi padronizada por meio de ontologias para representar
e organizar dito conhecimento, conseguindo assim uma representação
compreensível pelos humanos e computadores \citep{allemang2011semantic},
além de fornecer suporte com outras tecnologias da web semântica e
assim realizar consultas complexas que permitam responder perguntas
de interesse para os usuários do sistema software.

O conhecimento sobre sustentabilidade no sistema de produção de cana-de-açúcar
foi representado por meio de entidades, classes, relações semânticas
e axiomas. Ditos elementos constituíram a ontologia, representado
formalmente os conceitos do domínio, os quais foram integrados em
cada uma das funcionalidades do sistema permitindo a personalização
e vinculação da informação para satisfazer os requisitos dos usuários
do sistema SustenAgro.

O desenvolvimento da Ontologia de Domínio do SustenAgro foi iniciado
com a criação de um mapa conceitual entre um grupo de especialistas
em modelagem de conhecimento. Na reunião da equipe na Embrapa Informática
Agropecuária (UNICAMP - Campinas), foram identificados os principais
conceitos em cada uma das dimensões da sustentabilidade: ambiental,
social e econômica. 

O sistemas agricolas foram modelados por meio de três subsistemas:
(i) o subsistema ambiental que fornece as condições físicas, químicas
e biológicas que suportam o desenvolvimento das culturas, (ii) o subsistema
social que integra organizações e pessoas que realizam a produção,
relacionando-se internamente e externamente com os sistemas produtivos
e (iii) o subsistema econômico que estabelece as condiciones de oferta
e demanda dos produtos e subprodutos do sistema de produção agrícola;
das interações entre estes subsistemas, emerge um comportamento complexo
que requer uma abordagem holística e inter-relacionada para suportar
a tomada de decisões que garantam a sustentabilidade do sistema em
analise.

Cada uma das dimensões faz a função de \emph{container}. Neles estão
contidos os indicadores que foram validados como os mais relevantes
para as condições gerais das fazendas e usinas produtoras de cana-de-açúcar
no estado de São Paulo. Os indicadores têm uma relação de \emph{contains}
com os atributos e uma relação de \emph{considers} com os componentes
dos indicadores.

As três dimensões da sustentabilidade têm uma participação equitativa
no método de avaliação \citep{kraines2011system}. A Figura \ref{fig:environment}
representa a dimensão ambiental, modelo onde são definidos os seguintes
conceitos (\emph{containers}):

\begin{figure}
\begin{centering}
\includegraphics[width=1\textwidth]{\string"/media/john/Data/Master Degree/Dissertation Document/figures/ambiental\string".eps}
\par\end{centering}
\caption{Mapa conceitual - Ambiental\label{fig:environment}}
\end{figure}

\begin{itemize}
\item Atributo solo: indicadores que avaliam os aspectos referentes às características
do solo.
\item Atributo hídrico: indicadores que avaliam os aspectos referentes à
disponibilidade e qualidade das fontes hídricas.
\item Atributo clima: indicadores que avaliam os aspectos climáticos.
\end{itemize}
Nesta dimensão (ambiental), não foi possível identificar indicadores
de tipo hídrico porque não existe consenso entre os especialistas
consultados sobre quais são os aspectos mais relevantes destes para
a avaliação da sustentabilidade, mas é um aspecto fundamental para
trabalhar nas próximas etapas de pesquisa.

A Figura \ref{fig:social}, representa a dimensão social, onde são
definidos os seguintes conceitos (\emph{containers}):

\begin{figure}
\centering{}\includegraphics[width=1\textwidth]{\string"/media/john/Data/Master Degree/Dissertation Document/figures/social\string".eps}\caption{Mapa conceitual - Social\label{fig:social}}
\end{figure}

\begin{itemize}
\item Atributo emprego e renda: indicadores que avaliam os aspectos referentes
à mão-de-obra.
\item Atributo saúde: indicadores que avaliam os aspectos de segurança dos
trabalhadores.
\item Atributo treinamento: indicadores que avaliam os aspectos da capacitação
dos trabalhadores.
\end{itemize}
Nesta dimensão (Social), é importante reconhecer que as unidades produtivas,
sejam do tipo fazendas ou usinas, são compostas por pessoas tanto
internamente como externamente. Por isso, é importante refinar os
indicadores para incluir a população externa à unidade produtiva que
é afetada pelas práticas produtivas.

As Figuras \ref{fig:Economic-1} e \ref{fig:Economic-2} apresentam
a dimensão econômica, onde foram definidos os seguintes conceitos
(\emph{containers}):

\begin{figure}
\begin{centering}
\includegraphics[width=1\textwidth]{\string"/media/john/Data/Master Degree/Dissertation Document/figures/economica_1\string".eps}
\par\end{centering}
\caption{Mapa conceitual - Dimensão Econômica primeira parte.\label{fig:Economic-1}}
\end{figure}

\begin{figure}
\begin{centering}
\includegraphics[width=1\textwidth]{\string"/media/john/Data/Master Degree/Dissertation Document/figures/economica_2\string".eps}
\par\end{centering}
\caption{Mapa conceitual - Dimensão Econômica segunda parte.\label{fig:Economic-2}}
\end{figure}

\begin{itemize}
\item Atributo industrial: indicadores que avaliam os aspectos industriais. 
\item Atributo área recuperada: indicadores que avaliam os aspectos da área
produtiva e das técnicas produtivas.
\item Atributo produtividade: indicadores que avaliam os aspectos dos produtos
e dos processos produtivos.
\item Atributo custo: indicadores que avaliam os aspectos dos custos da
produção. 
\end{itemize}
Cada uma das três dimensões devem ser avaliadas equitativamente para
gerar um resultado coerente com a teoria da sustentabilidade agrícola.

A Figura \ref{fig:Method} mostra os conceitos que fazem a união das
dimensões e do método de avaliação. Cada um dos conceitos relacionados
com o método de avaliação utilizam os indicadores para realizar o
processo de avaliação. A intenção é representar o mais detalhadamente
e claramente possível o processo de avaliação para a sus correta execução.

\begin{figure}
\begin{centering}
\includegraphics[width=1\textwidth]{\string"/media/john/Data/Master Degree/Dissertation Document/figures/metodo\string".eps}
\par\end{centering}
\caption{Mapa conceitual - Método\label{fig:Method}}
\end{figure}


\section{User Stories}

Histórias de usuário são uma técnica para descrever, de uma forma
curta e simples, as características do sistema a partir da perspectiva
do usuário ou cliente do sistema, gerando uma definição de alto nível
de um requisito. Seu padrão é: Como um “tipo de usuário”, eu quero
“algum objetivo” para “alguma finalidade”.

Na aplicação dessa técnica foram obtidos as seguintes histórias:
\begin{enumerate}
\item O usuário poderá identificar e cadastrar a localização geográfica
e a área da sua lavoura (definir região geográfica do IBGE, latitude
e longitude - a partir do Google Maps). 
\item O usuário poderá identificar e cadastrar a microrregião a que pertence
a sua lavoura. O sistema fará uma sugestão de cadastro a partir dos
dados da localização geográfica.
\item O usuário deverá preencher o estado de cada indicador específico nas
dimensões ambiental, econômica e social. Esses indicadores vão ser
definidos pelo programa. Eles devem se adaptar às condições das regiões
e microrregiões do Brasil. Da mesma forma as faixas de limiares de
sustentabilidade são definidas.
\item Permitir o emprego da metodologia para avaliação caso a caso: possibilitar
que o usuário selecione quais indicadores vai utilizar. Dentro dos
indicadores, ele pode recomendar limiares mais adequados para a sua
realidade. Ele também pode inserir novos indicadores / limiares.
\item O usuário poderá obter o resultado dos índices segundo a informação
preenchida e a formula de agregação dos indicadores.
\item O usuário poderá armazenar a informação dos indicadores para futuras
consultas.
\item O usuário poderá acrescentar indicadores que considere importantes
para sua análise. Devem-se estabelecer regras para essa funcionalidade
de tal modo que os novos indicadores (criados pelos usuários) sejam
recuperáveis de um modo separado dos indicadores cadastrados no sistema. 
\item Cronograma de avaliação, melhor depois de cada safra. 
\end{enumerate}
O usuário deverá ser informado da importância dos processos de avaliação,
exemplo: 
\begin{itemize}
\item “A crescente demanda de países desenvolvidos por produtos com garantia
de origem tem induzido aumento das certificações nas usinas no Brasil
(ALVES et al., 2008).” 
\item A certificação tem sido uma importante forma de diferenciação de commodities
agrícolas, facilitando seu acesso aos mercados protegidos dos países
desenvolvidos. 
\item A caracterização climática aliada aos detalhes de fertilidade e manejo
do solo (quantificação edafoclimática) são essenciais para a determinação
das regiões aptas ao cultivo de culturas de interesse comercial (CIIAGRO,
2009). 
\end{itemize}
Depois do ingresso da informação sobre os indicadores, o usuário receberá
recomendações classificadas sobre práticas de sustentabilidade recomendadas
com sua argumentação, exemplo: 
\begin{itemize}
\item (Ambiental) “O sistema de plantio direto da cana-de-açúcar sobre leguminosas
proporciona maiores teores foliares de N e K na cana do que o plantio
convencional (JÚNIOR; COELHO, 2008)”.
\item (Ambiental) Segundo Leme (2005), haveria redução de 36\% na emissão
de gases do efeito estufa (GEE) se a palha fosse queimada nas caldeiras
das usinas e destilarias, ao invés de ser queimada no campo.
\item (Ambiental) A queima da cana aumenta a erosão do solo e a poluição
do ar e reduz a qualidade da matéria-prima (LINS; SAAVEDR, 2007). 
\item (Ambiental) Quando a cana não é queimada, proliferam, nos canaviais,
roedores silvestres originários de fragmentos florestais. Esses roedores
podem transmitir o Hantavírus através da urina e contaminar cortadores
de cana, causando uma síndrome respiratória e cardíaca, a pneumocitose,
podendo levar à morte. 
\item (Ambiental) Quando não há queima da cana é comum, também, o aumento
do ataque de cigarrinhas, com perdas significativas de produção (ANDRADE;
DINIZ, 2007). 
\item (Econômico) A utilização das colheitadeiras reverte-se em aumento
da produtividade e da qualidade da matéria-prima, bem como em diminuição
dos custos da produção agrícola, que representam entre 50\% e 60\%
em relação ao custo total (SCOPINHO, 1995).
\item (Econômico e Social) A utilização das colheitadeiras em cooperativa
possibilita a soma das áreas de produtores próximos possibilitando
a mecanização em propriedades com restrição para mecanização.
\item (Econômico) Restrições físicas da propriedade (menos de 500 ha de
área com declividade inferior a 12\% e talhões menores que 800 metros)
dificultam a mecanização. 
\end{itemize}

\section{Scenarios}

É uma técnica que permite a descrição das funcionalidades do sistema
da perspectiva do usuário ou cliente com a descrição detalhada da
interação destes. Em geral, é uma descrição detalhada de cada um dos
passos dos usuários no sistema para alcançar seu objetivo. Abaixo,
serão apresentadas as 8 histórias de usuários do projeto SustenAgro
com os cenários associados a elas:

\textbf{História de usuário \#1:} “O usuário poderá identificar e
cadastrar a localização geográfica e a área da sua lavoura (definir
região geográfica do IBGE, latitude e longitude - a partir do Google
Maps).”
\begin{enumerate}
\item O usuário ingressa na sua conta, através do sistema web SustenAgro
em \url{http://sustenagro.embrapa.br}, e o sistema apresenta a tela
“Home” 
\item O usuário seleciona a aba “lavouras” e dá um click em ``cadastrar
lavoura''. O sistema apresenta a tela de cadastro de lavouras, onde
tem um mapa do Google Maps 
\item O usuário seleciona no mapa um ponto que identificará a localização
da lavoura. Se ele quiser, também é possível marcar a área da lavoura
para que o sistema possa ter dados mais específicos para o processo
de avaliação de sustentabilidade. Uma vez terminado, o usuário dá
um click no botão “seguinte” e o sistema cadastra a informação preenchida. 
\end{enumerate}
\textbf{História de usuário \#2}: “O usuário poderá identificar e
cadastrar a microrregião a que pertence a sua lavoura por meio de
uma sugestão que o sistema faz com os dados da localização geográfica.”
\begin{enumerate}
\item O usuário poderá fazer a “Historia de usuário \#1” ou entrar no sistema
e continuar com o cadastro da lavoura de onde ele tenha parado. O
sistema apresentará uma tela com sugestões de microrregiões. 
\item O usuário poderá escolher a microrregião onde esteja localizada a
lavoura e salvá-la no sistema por meio do botão ``seguinte''. 
\end{enumerate}
\textbf{História de usuário \#3:} “O usuário deverá preencher o estado
de cada indicador específico nas dimensões ambiental, econômica e
social. Esses indicadores vão ser definidos pelo programa. Eles devem
se adaptar às condições das regiões e microrregiões do Brasil. Da
mesma forma as faixas de limiares de sustentabilidade são definidas.''
\begin{enumerate}
\item O usuário poderá fazer a “História de usuário \#2” ou entrar no sistema
e continuar com o cadastro dos indicadores de onde ele tenha parado.
O sistema apresentará uma tela com três abas que contém os controles
que permitiram fazer o cadastro dos indicadores nas dimensões ambiental,
econômica e social. 
\item O usuário dá um click na primeira aba e começa a preencher os dados
dos indicadores ambientais, principalmente os limiares que identificam
o estado do indicador. A interface também permite eliminar ou acrescentar
indicadores específicos por parte dos usuários (funcionalidade que
é explicada na “história de usuário \#4”). 
\item O usuário preenche os dados das outras duas dimensões e o sistema
salva as mudanças.
\end{enumerate}
\textbf{História de usuário \#4:} “Permitir o emprego da metodologia
para avaliação caso a caso: possibilitar que o usuário selecione quais
indicadores vai utilizar. Dentro dos indicadores, ele pode recomendar
limiares mais adequados para a sua realidade. Ele também pode inserir
novos indicadores\slash{}limiares.”
\begin{enumerate}
\item O usuário poderá fazer a “Historia de usuário \#3” ou entrar no sistema
e continuar na tela de cadastro de indicadores e, quando aconteça
que o usuário precise de um indicador que não seja oferecido pelo
sistema, o usuário poderá acrescentá-lo por meio do botão “acrescentar
indicador” 
\item O usuário da click no botão “acrescentar indicador” e lhe é apresentada
uma interface de entrada, onde ele deverá cadastrar o título, a descrição,
os limiares, a medida do manejo e a justificativa desse indicador.
Depois, preenche o estado do indicador e o sistema salva esses dados
nessa dimensão. 
\item O usuário também poderá eliminar alguns indicadores segundo seu critério.
\end{enumerate}
\textbf{História de usuário \#5:} \textquotedbl{}O usuário poderá
obter o resultado dos índices segundo a informação preenchida e a
formula de agregação dos indicadores.\textquotedbl{}
\begin{enumerate}
\item Depois de terminada a “História de usuário \#4”, o sistema fará a
aplicação da metodologia de avaliação, que vai estar definida no sistema
pelos administradores. 
\item O resultado da avaliação vai ser cadastrado no sistema com informações
sobre a metodologia utilizada.
\item A metodologia de avaliação pode ser atualizada pelos administradores
para ser utilizada em avaliações futuras.
\end{enumerate}
\textbf{História de usuário \#6:} ``O usuário poderá armazenar a
informação dos indicadores para futuras consultas.''
\begin{enumerate}
\item O usuário faz qualquer tipo de entrada de dados nos formulários do
SustenAgro. 
\item Esses dados vão ser salvos quando o usuário mudar de formulário ou
quando der um click no botão ``seguinte''.
\end{enumerate}
\textbf{História de usuário \#7:} ``O usuário poderá acrescentar
indicadores que considere importantes para sua análise. Devem-se estabelecer
regras para essa funcionalidade de tal modo que os novos indicadores
(criados pelos usuários) sejam recuperáveis de um modo separado dos
indicadores cadastrados no sistema.''
\begin{enumerate}
\item Quando o usuário estiver preenchendo os indicadores gerados pelo sistema,
o sistema fornecerá um conjunto de controles que permitam a inclusão
de um novo indicador. Esse novo indicador vai ser definido pelo próprio
usuário baseado na sua experiência na área. 
\item O sistema armazenará esse novo indicador com uma classificação especial
que permita sua identificação para avaliar sua relevância. 
\item O usuário poderá preencher os dados do novo indicador, para que sejam
inclusos na avaliação de sustentabilidade.
\end{enumerate}
\textbf{História de usuário \#8:} ``Cronograma de avaliação, melhor
depois de cada safra.''
\begin{enumerate}
\item Depois de fazer o cadastro da fazenda e das culturas que são plantadas
nela, o sistema poderá identificar quando termina cada safra, gerando
um alerta para que u usuário faça o processo de avaliação nessa data.
\item O usuário lerá o alerta e poderá fazer o processo de avaliação de
sustentabilidade. 
\end{enumerate}

\section{Storyboard}

Storyboards são similares aos cenários. Elas ilustram a interação
necessária para se atingir um objetivo sem utilizar uma lista de passos,
a interação é visualizada por meio de uma história de quadrinhos.

Esta representação permite se ter uma visão holística da interação
do usuário, com ênfase nos aspectos funcionais da interação e não
nos aspectos da interface de usuário. A seguir, são apresentados os
textos das storyboard dos processos identificados:

\begin{figure}[H]
\begin{centering}
\includegraphics[width=1\columnwidth]{\string"/media/john/Data/Master Degree/Dissertation Document/figures/Story_1\string".eps}
\par\end{centering}
\begin{centering}
\emph{\small{}StoryBoard 1.}
\par\end{centering}{\small \par}
\smallskip{}

\begin{centering}
\includegraphics[width=1\columnwidth]{\string"/media/john/Data/Master Degree/Dissertation Document/figures/Story_2\string".eps}
\par\end{centering}
\begin{centering}
\textit{\small{}StoryBoard 2.}
\par\end{centering}{\small \par}
\smallskip{}

\begin{centering}
\includegraphics[width=1\columnwidth]{\string"/media/john/Data/Master Degree/Dissertation Document/figures/Story_3\string".eps}
\par\end{centering}
\begin{centering}
\textit{\small{}StoryBoard 3.}
\par\end{centering}{\small \par}
\centering{}\caption{Storyboards números 1–3.\label{fig:Storyboard-numero-1}}
\end{figure}

\begin{figure}[H]
\begin{centering}
\includegraphics[width=1\columnwidth]{\string"/media/john/Data/Master Degree/Dissertation Document/figures/Story_4\string".eps}
\par\end{centering}
\begin{centering}
\textit{\small{}StoryBoard 4.}
\par\end{centering}{\small \par}
\bigskip{}

\begin{centering}
\includegraphics[width=1\columnwidth]{\string"/media/john/Data/Master Degree/Dissertation Document/figures/Story_5\string".eps}
\par\end{centering}
\begin{centering}
\textit{\small{}StoryBoard 5.}
\par\end{centering}{\small \par}
\bigskip{}

\begin{centering}
\includegraphics[width=1\columnwidth]{\string"/media/john/Data/Master Degree/Dissertation Document/figures/Story_7\string".eps}
\par\end{centering}
\begin{centering}
\textit{\small{}StoryBoard 6.}
\par\end{centering}{\small \par}
\bigskip{}

\centering{}\caption{Storyboards números 4–6.\label{fig:Storyboard-numero-4}}
\end{figure}


\section{Mockups das Interfaces do SustenAgro}

Mockups permitem uma representação visual das interfaces do sistema
para ajudar no seu entendimento, fazer demonstrações, avaliações do
design, dentre outros propósitos. As Figuras \ref{fig:Mockup_home}
e \ref{fig:Mockup_indicators} mostram algumas telas com desenhos
dos Mockups que foram avaliados e validados pela equipe do projeto.

\begin{figure}
\centering{}\includegraphics[width=1\columnwidth]{\string"/media/john/Data/Master Degree/Dissertation Document/figures/Mockup Main\string".eps}\caption{Mockup da tela da Home Page do SustenAgro.\label{fig:Mockup_home}}
\end{figure}

\begin{figure}
\centering{}\includegraphics[width=1\columnwidth]{\string"/media/john/Data/Master Degree/Dissertation Document/figures/Tool_environmental_indicators\string".eps}\caption{Mockup da tela de indicadores do SustenAgro.\label{fig:Mockup_indicators}}
\end{figure}


\section{Protótipo da Interface Gráfica do SustenAgro}

O primeiro protótipo da interface gráfica do SustenAgro está publicado
nos servidores do laboratório Intermídia do ICMC\nobreakdash-USP
\footnote{http://biomac.icmc.usp.br:8080/sustenagro/}, na Figura
\ref{fig:Home} é apresentada a página inicial do protótipo.

\begin{figure}
\begin{centering}
\includegraphics[width=1\textwidth]{\string"/media/john/Data/Master Degree/Dissertation Document/figures/home\string".eps}
\par\end{centering}
\caption{Protótipo do SustenAgro – Home Page.\label{fig:Home}}
\end{figure}

Nessa tela pode-se observar o texto explicativo da ferramenta e as
abas de ``Início'', ``Ferramenta'' e ``Contato''. O menu da
ferramenta permite iniciar o processo de avaliação de sustentabilidade.

Na Figura \ref{fig:Indicators}, é apresentada a página dos indicadores,
onde se descreve o processo de avaliação. Ele começa com uma descrição
base do processo, a localização geográfica da unidade produtiva, a
caracterização dela, os indicadores e as recomendações que o sistema
vai gerar.

\begin{figure}
\centering{}\includegraphics[width=1\columnwidth]{\string"/media/john/Data/Master Degree/Dissertation Document/figures/indicators\string".eps}\caption{Protótipo do SustenAgro - Indicadores.\label{fig:Indicators}}
\end{figure}


\chapter{Decisioner: Sistema gerador de SADs}

Neste capítulo será descrito o desenvolvimento tecnologico realizado
no mestrado e que deu como resultado o Sistema SustenAgro, primeiramente
apresenta-se a metodologia usada.

O conhecimento sobre sustentabilidade no sistema de produção de cana-de-açúcar
foi representado por meio de entidades, classes, relações semânticas
e axiomas. Ditos elementos constituíram a ontologia, representado
formalmente os conceitos do domínio, os quais foram integrados em
cada uma das funcionalidades do sistema permitindo a personalização
e vinculação da informação para satisfazer os requisitos dos usuários
do sistema SustenAgro.

\section{Metodologia}

Os componentes da arquitetura do sistema web do SustenAgro são parte
deste trabalho (interface gráfica) e parte de um outro trabalho de
mestrado. A ideia é construir componentes que possam ser reusados
em outros SADs que trabalhem em domínios similares ao SustenAgro.
A equipe do SustenAgro testará os conceitos deste trabalho através
da avaliação de protótipos.

O desenvolvimento do SustenAgro será feito usando-se uma DSL baseada
na linguagem Groovy \citep{koenig2007groovy}. Ou seja, essa DSL será
uma extensão da linguagem Groovy. Groovy é uma linguagem que tem suporte
ao desenvolvimento de DSLs. Isso inclui suporte a DSL Descriptors,
arquivos Groovy que descrevem extensões \emph{domain-specific} para
o motor de inferência e assistente de conteúdo do plugin Groovy\nobreakdash-Eclipse.
Isso permite que a DSL criada tenha todo o mesmo suporte que o IDE\nomenclature{IDE}{Integrated Development Environment}
Eclipse dá a linguagens como Java ou Groovy, como code completion,
debugging, etc. Uma outra vantagem de Groovy é a disponibilidade do
Grails Framework para a criação de aplicações Web \citep{judd2008beginning}.
O desenvolvimento dessa DSL será feito em outro trabalho de mestrado.
Mas este trabalho irá contribuir com a parte da DSL que tem haver
com interfaces, além da ontologia de Controles Gráficos.

O uso da DSL por especialistas em sustentabilidade deve diminuir o
esforço necessário para se desenvolver um SAD nesse domínio. Mas mesmo
assim, ainda será necessário aplicar alguma metodologia de desenvolvimento
de software. 

Existem múltiplos métodos e metodologias que permitem um desenvolvimento
ágil de software. Nesse contexto, o termo ágil refere-se ao desenvolvimento
em tempos curtos e geração de protótipos facilmente adaptáveis às
mudanças. Exemplos de métodos ágeis são: “Mockups”, “User Stories”,
“Scenarios”, “Storyboards” e “Use Cases”, exemplos de metodologias
ágeis são: “SCRUM” ou “XP eXtreme Programming”.

Uma das etapas mais importantes dos desenvolvimentos ágeis é o levantamento
de requisitos. Essa etapa tem como objetivo definir as características
do software e pode ser realizada múltiplas vezes. Isso ocorre pois
as metodologias ágeis são cíclicas e os protótipos mudam em cada ciclo
para cumprir os requisitos.

O desenvolvimento do sistema SustenAgro será realizado por meio de
metodologias ágeis de desenvolvimento de software, principalmente
serão utilizadas algumas práticas da metodologia SCRUM \citep{schwaber2002gile}.
Também será usado o enfoque User-Centered Design. Nesse sentido, está
sendo desenvolvido primeiramente um \emph{mockup} da interface gráfica
do sistema, o qual será o meio de comunicação com os parceiros do
SustenAgro para determinar as funcionalidades básicas do sistema.
Quando o \emph{mockup} for validado, será iniciado o desenvolvimento
de um protótipo da interface gráfica que permitirá determinar os requisitos
funcionais.

Baseando-se na DSL, pode-se suportar um sistema gerador de interfaces
gráficas para conceder usabilidade e flexibilidade ao sistema. Essa
última característica constitui uma nova proposta de desenvolvimento
de SADs que permite a adaptação automática (ou semi\nobreakdash-automática)
da interface às mudanças dos conceitos do domínio.

Cada vez que sejam desenvolvidos cada componente de SustenAgro se
realizarão diversos testes para validar as funcionalidades do sistema,
esse processo será realizado com os especialistas para refinar as
funcionalidades do sistema de acordo com os requisitos manifestados.
Espera-se que, usando a DSL, os próprios especialistas vão ser capazes
de fazer parte do desenvolvimento e validação.

\begin{comment}
\begin{enumerate}
\item \textbf{!!Ainda falta mexer com esta parte!!} 
\item Recuperação dos dados dos repositórios FAO Linked Data\nomenclature{FAO Linked Data}{Food and Agriculture Organization of the United Nations (FAO) Linked Data}
e AGRIS referente aos locais que foram realizadas coletas de espécimes

\begin{enumerate}
\item Análise dos dados de baixa qualidade, ou seja, dados que não tem informações
importantes, por exemplo, localidade e município
\item Verificação do número de dados imprecisos e exibição dos mesmos em
um mapa
\item Verificação do número de dados que contém informações sobre local,
latitude, longitude
\item Verificação da quantidade de dados que possuem informações de latitude
e longitude antes e depois do uso de GPS por biólogos em coletas
\end{enumerate}
\item Utilização de dados de fontes externas como Geonames, Wikimapia, DBpedia
\item Implementação de um método para agrupar todos os dados dos repositórios
SpeciesLink e GBIF e realizar a resolução de topônimos utilizando
técnicas de Recuperação de Informação Geográfica e ontologias
\item Aprimoramento das informações geográficas ausentes nos dados do SpeciesLink
e GBIF

\begin{enumerate}
\item Utilização dos dados referente aos repositórios externos para melhorar
os registros das localidades dos repositórios SpeciesLink e GBIF
\item Criação um método para sumarizar as coordenadas geográficas de acordo
com a abordagem da Lei de Linus
\end{enumerate}
\item Verificação da quantidade de dados que tiveram suas informações aprimoradas

\begin{enumerate}
\item Contagem dos dados que não contém informações sobre latitude, longitude
e foram recuperados
\item Contagem dos dados que possui informações geográficas recuperadas
e eram muito velhos, ou seja, antes do uso de GPS por biólogos em
coletas
\item Análise dos resultados dos passos a) e b) anteriores utilizando o
teste t de Student, para verificar o quão boa foi a abordagem utilizada
\end{enumerate}
\item Mapeamento dos dados para uma \textit{triplestore} utilizando ontologias

\begin{enumerate}
\item Definição das triplas em RDF, que deveram possuir um sujeito, predicado
e objeto para cada localidade. 
\item Mapeamento de coordenadas geográficas para GeoSPARQL 
\end{enumerate}
\item Construção de uma base de teste com informações sobre qual consulta
representa uma localidade

\begin{enumerate}
\item Construção de consultas semânticas e verificação de resultados utilizando
as medidas de precisão, revocação e F1 
\item Realização da mesma abordagem do passo a) para os repositórios DBpedia
e Geonames, com intuito de verificar a viabilidade para expansão de
consultas
\end{enumerate}
\item Desenvolvimento de uma interface que permita aos biólogos inserir
dados no \textit{Gazetteer }por meio de mapas

\begin{enumerate}
\item Desenvolvimento de um método que permita aos biólogos inserir lugares
usando GeoTAGS
\item Agrupamento dos dados inseridos e aprimoramento dos dados referentes
a localidades que são similares.
\item Desenvolvimento de um método que permita aos biólogos inserirem links
do DBpedia, quando os mesmos existirem, para auxiliar no crescimento
da \textit{Web of Data}. 
\item Desenvolvimento de um método que permita aos biólogos verificarem
a acurácia dos links sobre a DBpedia inseridos no \textit{Gazetteer} 
\end{enumerate}
\item Verificação do número de lugares inseridos pelos usuários e qualidade
dos dados 

\begin{enumerate}
\item Verificação da média de usuários que concordam com as coordenadas
geográficas
\item Verificação da média de usuários que concordam com as informações
de \emph{Linked Open Data} presentes no \textit{Gazetteer} 
\item Verificação da quantidade de dados que tiveram suas informações aprimoradas
\end{enumerate}
\end{enumerate}
\end{comment}


\subsection{Atividades Concluídas até o Momento}

Quanto a metodologia proposta para desenvolvimento do sistema, os
passos 1 ao 9 já foram concluídos, necessitando apenas alguns ajustes
e integração das novas funcionalidades que serão implementadas no
passo 10. No cronograma, todas as atividades de A1 a A6 foram concluídas.
Além disso, a redação e submissão de artigos com os resultados obtidos,
estão sendo realizadas.

\section{Dificuldades e Limitações}

Até o presente momento, foi evidenciado como dificuldade para desenvolvimento
do projeto as escassas fontes de informação que forneçam uma conexão
entre sistemas de produção agrícola e sustentabilidade. Só foi possível
encontrar fontes de informação especializada em cada área do conhecimento
de maneira separada. Outro problema é a falta de dados resultantes
da aplicação dos indicadores de sustentabilidade fornecidos pela Embrapa.

\section{Sistemas de apóio à decisão}

Os sistemas de apóio à decisão (SAD) ajudam no entendimento de processos
complexos, auxiliam na comparação dos fenômenos envolvidos e suportam
a análise e escolha de alternativas no processo de decisão \citep{heinzle2010semantica}.

O sistema SustenAgro é um SAD e será desenvolvido com o apoio da equipe
do projeto SustenAgro (Anexo \ref{chap:Projeto-SustenAgro}) da Embrapa
Meio Ambiente, a qual está desenvolvendo uma proposta metodológica
para avaliar a sustentabilidade de sistemas de produção de cana-de-açúcar
no Centro Sul do Brasil para equacionar as principais questões referentes
a esses sistemas produtivos e possibilitar a utilização racional dos
recursos naturais para suprir as necessidades presentes e garantir
o suprimento das gerações futuras.

A equipe de TI do SustenAgro determinou que o tipo de sistema mais
conveniente para o desenvolvimento seria um Sistema de Apóio à Decisão
(SAD). Com a finalidade de definir a arquitetura e a interface gráfica
desse sistema realizaram-se duas perguntas de pesquisa que orientaram
esse projeto:
\begin{itemize}
\item Como integrar o conhecimento dos especialistas em um sistema de apoio
na tomada de decisões permitindo a continua mudança do modelo do domínio?
\item Como gerar interfaces gráficas a partir de definições simples do domínio
do conhecimento?
\end{itemize}
Tendo em conta os requisitos do software, como o suporte a contínua
mudança do modelo de dados e a geração dinâmica de interfaces, se
propõe a arquitetura a seguir.

\section{Arquitetura do Sistema}

O sistema SustenAgro será composto por vários componentes. A representação
da arquitetura do sistema é apresentada na figura \ref{fig:Architecture},
a qual contém os seguintes elementos:

\begin{figure}[H]
\centering{}\includegraphics[width=0.9\columnwidth]{\string"/media/john/Data/Master Degree/Dissertation Document/figures/arquitecture\string".eps}\caption{Arquitetura de SustenAgro\label{fig:Architecture}}
\end{figure}

\begin{enumerate}
\item Ontologia do domínio: Ontologia que vai representar os conceitos do
domínio: avaliação da sustentabilidade do sistema produtivo de cana-de-açúcar.
Ela é a base fundamental para o sistema SustenAgro porque permite
estabelecer os conceitos fundamentais que vão ser utilizados pelo
sistema, entre eles: indicadores, componentes de indicadores, índices,
dimensões da sustentabilidade, recomendações e o método de avaliação.
\item TripleStore: Sistema de recuperação da informação que permitirá padronizar
as informações em formato de triplas, permitindo a compatibilidade
e o reúso das informações entre fontes de dados externas.
\item Ontologia de Controles Gráficos: Ontologia que representará os controles
de usuários. Ela tem a finalidade de permitir a manipulação desses
controles por meio de uma DSL. Ela vai representar cada um dos tipos
de controles e suas funcionalidades e fazer um mapeamento deles com
os tipos de dados da ontologia de domínio. 
\item DSL de Interfaces: Linguagem especifica do domínio dos controles web
que serão usados pelo SustenAgro. Ela permitirá uma definição flexível
das interfaces, baseada nos conceitos definidos na ontologia de domínio
e de controles gráficos. Ela permitirá a definição das características
visuais e dos tipos de controles especializados para cada conceito
da ontologia de domínio.
\item Sistema Gerador de Interfaces Gráficas: Sistema no navegador de internet
\emph{(browser)} que cria uma interface a partir da DSL e da ontologia
de controles gráficos.
\end{enumerate}
Os componentes da arquitetura do SustenAgro são parte deste trabalho
(interface gráfica) e parte de outro trabalho de mestrado. Esses componentes
não serão exclusivos do SustenAgro, podendo ser reusados em outros
SADs. O SustenAgro e sua equipe testarão os conceitos deste trabalho
através de protótipos.

\section{Decisioner: gerador de sistemas de avaliação}

O sistema gerador de interfaces é uma camada adicional ao processo
de definição da interface gráfica. Ele usa a DSL de Interface e a
as ontologias (de domínio e da UI), Figura \ref{fig:interfaces},
para gerar a interface Web no padrão HTML. A Figura \ref{fig:interfaces}
apresentada a arquitetura do sistema como um todo.

\begin{figure}[H]
\centering{}\includegraphics[scale=0.5]{\string"/media/john/Data/Master Degree/Dissertation Document/figures/interfaces\string".eps}\caption{Processo de geração de interfaces gráficas\label{fig:interfaces}}
\end{figure}


\section{Considerações Finais}

O desenvolvimento do sistema Sustenagro satisfaz uma necessidade presente
na unidade da Embrapa Meio Ambiente: um sistema de avaliação de sustentabilidade
em sistemas produtivos de cana-de-açúcar. Ele permitirá adquirir dados
do estado atual de sustentabilidade nas fazendas e usinas e assim
embasar e formalizar políticas publicas para promover práticas produtivas
mais sustentáveis de acordo com critérios ambientais, sociais e econômicos.

Além de satisfazer uma necessidade institucional, o SustenAgro se
consolida como uma proposta de SAD baseado em conhecimento e vinculado
às tecnologias da web semântica, um processo que requer um trabalho
de pesquisa e de inovação tecnológica. A pesquisa deste trabalho de
mestrado, usará o SustenAgro como uma base de testes realista para
os conceitos e ferramentas desenvolvidos. 

Após o desenvolvimento do Sistema SustenAgro, poder-se-a analisar
as características fundamentais desse tipo de SAD e tentar reusar
a arquitetura em outros SADs da própria Embrapa.
