A construção de sistemas que sejam capazes de fornecer um suporte
ao gestor em um processo de tomada de decisões vem sendo um desafio
ao longo dos anos, os SADs são sistemas software que visam melhorar
a tomada de decisão individual ou grupal, combinando o conhecimento
do tomador(es) de decisão com dados relevantes de fontes confiáveis,
nos quais são aplicados métodos e modelos matemáticos para suportar
o analise, comparação e escolha de alternativas no processo de decisão
\citet{Tweedale2016} 

SADs auxiliam tomadores de decisão dando-lhes um maior entendimento
do domínio, eles combinam as habilidades e as metodologias dos especialistas
(humanos)\citet{heinzle2010semantica}, à capacidade dos computadores
de acessar dados, estruturar eles em modelos, interpretar, formular
e avaliar alternativas e cenários distintos onde podem haver possíveis
soluções para os problemas que se querem solucionar \citet{lu2006application}.

SADs são criados por especialistas nas áreas de domínio nas quais
eles serão aplicados e implementados por programadores, esse pode
ser um processo lento e custoso, já que os dois grupos de profissionais
têm \foreignlanguage{english}{\emph{Backgrounds}} diferentes e vão
ter problemas de comunicação durante o processo de criação e testes
de um SAD. Esses profissionais podem ser até de organizações diferentes,
o que dificulta ainda mais o processo.

\section{Arquitetura para Sistemas de Apoio à Decisão}

Durante a evolução dos SAD, tem acontecido varias melhoras na arquitetura
deles, a tendencia é a integração com os métodos de inteligencia artificial
para estender a aplicabilidade dos SAD a problemas complexos, gerando
os \foreignlanguage{english}{Intelligent Decision Support System (IDSS)}
que se caracterizam por ter agentes inteligentes, entre os métodos
usados estão as bases de conhecimento que suportam a inferência, permitindo
o desenvolvimento de \foreignlanguage{english}{Expert Systems} e \foreignlanguage{english}{Knowledge
Based Systems} sistemas são classificados como \foreignlanguage{english}{Rule
Based Systems}\citet{Tweedale2016}, este tipo de sistemas conformam
o escopo da pesquisa apresentada neste documento.

\begin{figure}[H]
\noindent \begin{centering}
\includegraphics[width=0.8\columnwidth]{\string"figures/Intelligent DSS\string".png}
\par\end{centering}
\caption{Componentes de um SAD .\label{fig:Componentes-SAD}}
\end{figure}

A Figura \ref{fig:Componentes-SAD} mostra os componentes de um SAD\citet{Tweedale2016},
que são descritos a continuação: 
\begin{description}
\item [{I\foreignlanguage{english}{nputs}}] Corresponde à entradas do sistema,
que esta composta do conhecimento dos especialistas que pode estar
representada em uma base de conhecimento, e dos dados que serão processados,
que podem estar em um banco de dados, estes dois componentes devem
ser o mais precisos e completos possíveis para garantir boas respostas
do sistema.
\selectlanguage{english}%
\item [{Processing}] \foreignlanguage{brazil}{Está composta pelos modelos
e métodos de organização e processamento dos dados, e restrições para
avaliar as alternativas de resposta, este componente contem os métodos
matemáticos que processam os dados, permitindo gerar os resultados
do sistema.}
\item [{Outputs}] \foreignlanguage{brazil}{São os resultados do processamento
dos inputs e permite comparar as alternativas de decisão, como saídas
comuns estão os relatórios, previsões e recomendações, os quais são
apresentados por meio de uma interface de usuário para facilitar o
entendimento e interação com o usuário}
\end{description}
Os SAD contem o conhecimento dos especialistas implícito nos componentes
dele, uma das mudanças da arquitetura realizada no presente trabalho,
foi que o conhecimento do especialista ficasse agrupado em uma ontologia
que permite definir, classificar, relacionar e inferir conhecimento,
com a finalidade de facilitar a definição e atualização de conceitos
por parte dos especialistas, permitindo que eles mesmos descrevam
o domínio sem precisar de desenvolvedores, e a partir dessa definição
computável será gerado o SAD, os especialistas desse domínio terão
familiaridade com os termos da ontologia e poderão especificar grande
parte conhecimento envolvido no SAD. Idealmente, essa definição deve
ser detalhada o suficiente para que os desenvolvedores possam desenvolver
a parte computacional do SAD sem necessidade de \foreignlanguage{english}{feedback}
dos especialistas.

\section{SAD SustenAgro}

O SAD Sustenagro requerido pelos especialistas em sustentabilidade
da Embrapa Meio Ambiente requiriu inicialmente as seguintes características:
\begin{itemize}
\item Sistema web com banco de dados que armazene e recupere as informações
do sistema.
\item Integração e implementação do método SustenAgro de avaliação da sustentabilidade
descrito no apêndice \ref{chap:Sustainability_Assessment} 
\item Suporte para adaptar o método SustenAgro a outras culturas.
\item Integração com o sistemas de georreferenciação.
\item Desenvolvimento de \foreignlanguage{english}{widgets} especificas
\foreignlanguage{english}{Sustainability Matrix }e \foreignlanguage{english}{Sustainability
Semaphore} 
\item Geração de relatórios e de recomendações de sustentabilidade.
\end{itemize}
\begin{figure}[h]
\begin{centering}
\includegraphics[width=0.6\columnwidth]{\string"figures/SustenAgro Initial Architecture\string".png}
\par\end{centering}
\caption{SustenAgro arquitetura inicial\label{fig:SustenAgro-arquitetura-inicial}}
\end{figure}

A figura \ref{fig:SustenAgro-arquitetura-inicial} mostra a arquitetura
inicial do SAD SustenAgro, a qual corresponde a um sistema de informação
tradicional que requer da intervenção dos desenvolvedores de software
para definir ou atualizar o conhecimento do domínio, exatamente o
problema que abordamos nesta pesquisa, pelo qual se tomo como projeto
piloto.

Um dos problemas identificados foi que os especialistas não tinham
uma definição clara do SAD SustenAgro, pelo que foi necessário realizar
um levantamento de requerimentos descrito no capitulo \ref{chap:SustenAgro}
para definir os requisitos funcionais e não funcionais, além disso
foi necessário reestruturar o desenvolvimento do SAD SustenAgro para
que faça parte do processo pesquisa.

\section{Trabalhos relacionados}

Com a finalidade de relacionar pesquisas de referencia que forneçam
ideias e exemplos para abordar o problema da seguinte pesquisa, realizou-se
uma consulta sobre, definição de SADs por meio de ontologias do domínio
dos especialistas, e SADs semelhantes ao SustenAgro, em diversas fontes
de informação acadêmica.

Sobre definição de SADs, encontrou-se que na literatura existem pesquisas
relacionadas com o vocabulário \foreignlanguage{english}{\emph{AGROVOC\nomenclature{AGROVOC}{Agricultural vocabulary}}}\emph{
}\foreignlanguage{english}{\emph{Agricultural Vocabulary}} \footnote{http://aims.fao.org/agrovoc}
que é um \foreignlanguage{english}{\emph{theasaurus}} que fornece
termos padronizados sobre alimentação, nutrição, agricultura, pesca,
floresta e meio ambiente criados de maneira colaborativa e coordenados
pela \emph{FAO}, estes termos podem ser reutilizados nas ontologias
\citep{DCMIPro841}, permitindo uma padronização com os identificadores
dos conceitos, reutilizando informações e integrando os conceitos
com outros dados da \foreignlanguage{english}{Linked Open Data} (\foreignlanguage{english}{LOD}\nomenclature{LOD}{Linked Open Data}),
esta reutilização foi feita através da vinculação de \foreignlanguage{english}{AGROVOC}
ao sistema \emph{AOS/CS }\foreignlanguage{english}{\emph{Agricultural
Ontology Service Concept Server}}, a FAO desenvolveu um modelo base
para esse novo sistema utilizando o \foreignlanguage{english}{\emph{OWL}}\emph{.}

\citet{kraines2011system} desenvolveu uma ferramenta com a visão
de criar um \foreignlanguage{english}{\emph{Knowledge Sharing System}}\emph{
for}\foreignlanguage{english}{\emph{ Sustainability Science}} por
meio do processo \foreignlanguage{english}{\emph{Semantic Data Modeling}},
uma ontologia fundamentada na lógica descritiva foi desenvolvida por
meio do ISO 15926 \foreignlanguage{english}{Data Model} para descrever
três tipos de conceitualizações de ciência sustentável: \foreignlanguage{english}{\emph{situational
knowledge}}, \foreignlanguage{english}{\emph{analytic methods}} e
\foreignlanguage{english}{\emph{scenario frameworks}}. Os conhecimentos
dos especialistas podem ser descritos por meio de \foreignlanguage{english}{\emph{semantic
statements}} utilizando dita ontologia o \foreignlanguage{english}{\emph{semantic
matching based on logic}} e \foreignlanguage{english}{\emph{rule-based
inferences}} foram usados para quantificar o \foreignlanguage{english}{\emph{conceptual
overlap of semantic statements}}.

Cada uma destas pesquisas fornece um exemplo do uso de ontologias
na criação de soluções baseadas em conhecimento, isto é confirmado
por \citet{roussey2010ontologies} por meio da descrição de como as
ontologias têm sido usadas para múltiplas tarefas, uma das quais é
conseguir interoperabilidade entre sistemas de informação heterogêneos
e como as seguintes gerações de sistemas de informação utilizariam
uma base do conhecimento do domínio, dadas as afirmações destas pesquisas
pode-se deduzir que uma ontologia proporciona o suporte conceitual
para cumprir os requisitos do sistema SustenAgro.

Sobre SADs semelhantes ao SustenAgro, encontrou-se que \citet{Wilson2007299},
analisaram e melhoraram a abordagem para o desenho e uso de modelos\emph{
}\foreignlanguage{english}{\emph{Sustainable Development Indicator}}
\emph{\nomenclature{SDI}{Sustainable Development Indicator}}, permitindo
avaliar se as métricas globais SDI mostrando uma abordagem clara para
definir métodos de avaliação o desenvolvimento sustentável. 

Seis métricas globais SDI são comparadas e os resultados ilustram
que as diferentes métricas variam à interpretação sobre a sustentabilidade
das nações, o grau de variabilidade entre as métricas é analisado
por meio de análises de correlação, ao final conclui-se que não existe
consenso sobre a melhor abordagem para definir métricas.

Uma estratégia para abordar a complexidade em SADs são métodos e metodologias
de avaliação, que utilizam indicadores, um exemplo desse enfoque é
a pesquisa de \citet{AlkanOlsson:2009}, nela foi desenvolvido um
\foreignlanguage{english}{\emph{framework}} de indicadores que relaciona,
de uma maneira consistente, as dimensões ambiental, econômica e social
do desenvolvimento sustentável. Seu principal benefício é uma relativa
simplicidade na apresentação da informação e a possibilidade de vincular
os indicadores com objetivos políticos de cada dimensão da sustentabilidade
e assim facilitar a comparação dos impactos das novas políticas em
cada dimensão.

Na pesquisa de \citet{Ewert2009546} apresenta varias estrategias
para abordar a complexidade nos sistemas agrícolas, começa relacionando
a agricultura com os sistemas socioeconômicos e naturais, e enfrenta
o problema de gerir suas múltiplas funções de uma maneira sustentável,
o método \foreignlanguage{english}{\emph{Integrated Assessment and
Modeling}}\emph{ (IAM)} pode fornecer uma visão sobre os possíveis
impactos das mudanças políticas, existem múltiplos modelos \foreignlanguage{english}{\emph{Integrated
Assessment}}\emph{ (IA)} mas a maioria são monolíticos resolvendo
problemas específicos, os enfoques flexíveis são escassos, o \emph{framework}
proposto (SEAMLESS-IF) integra relações e processos através de \emph{''disciplinas
e escalas''} e combina análises quantitativos com apreciações qualitativas
e experiências, permitindo um acoplamento entre modelos e ferramentas,
este \emph{framework} permite um avanço significativo em flexibilidade
de IAM o que permite melhorar a modelagem integrado para avaliação
do impacto em agricultura. A pesquisa apresenta exemplos da natureza
complexa do sistema agrícola.

A pesquisa feita por \citet{boseley2009development}, apresenta um
exemplo de como unificar termos de plantas e organizá-los em uma maneira
sistemática é fundamental para mais eficiência nas pesquisas e descobertas,
para este fim, a \foreignlanguage{english}{\emph{Plant Ontology}}
foi criada como uma iniciativa do \foreignlanguage{english}{\emph{Plant
Ontology Consortium}}, esta ontologia é um vocabulário controlado
de termos usados para dados de atributos (por exemplo, genótipo e
fenótipo) para uma estrutura especifica da planta ou um estagio de
desenvolvimento.

Existem pesquisas que abordam a sustentabilidade a través de ferramentas
tecnológicas, as quais podem servir de referencia ao sistema SustenAgro,
uma delas foi desenvolvida por \citet{brilhante:2006} e consiste
em um \emph{framework} (MOeMA-IS) para análise sustentável do estado
do Amazonas, que usa uma ontologia para descrição de indicadores de
sustentabilidade (\foreignlanguage{english}{ISD-Economics Ontology}),
onde são usados os indicadores humano (Social), suporte (Econômico)
e natural (Ambiental), os quais foram subdivididos em sete indicadores,
seu desenvolvimento foi feito de uma maneira genérica de forma que
ela suporta a inclusão de novos indicadores de forma simples, esta
ontologia foi feita em dois níveis de hierarquia, o \emph{framework
}trabalha de forma onde a base dele é a ontologia e ele emprega os
indicadores de bases de dados (não foi descrito se são \foreignlanguage{english}{\emph{triple-stores}})
e as medidas e valores padrões de outra base e assim o \emph{framework}
calcula as medias dos indicadores e dá um resultado relevante ao tipo
de necessidade. Para o desenvolvimento da ontologia foi utilizada
a ferramenta \emph{Protégé} utilizando o \foreignlanguage{english}{plug-in}
de \foreignlanguage{english}{OWL} e alguns indicadores foram feitos
utilizando a classe do SUMO do \foreignlanguage{english}{OWL}.

\section{Considerações finais}

O desenvolvimento do SAD SustenAgro, não satisfaz os requisitos de
desenvolvimento de um trabalho de mestrado, devido a que não envolve
a utilização de métodos e técnicas de investigação cientifica, pelo
qual será incluído dentro de um processo maior de pesquisa, onde seja
possível incluir o desenvolvimento dele com a finalidade de desenvolver
uma hipótese de pesquisa, a solução do problema do presente trabalho
``A inexistência de uma representação de conhecimento para definir
SADs, que tenha um formato computável, entendível e acessível pelos
especialistas do domínio e desenvolvedores de software'' requiriu
um direcionamento cientifico no desenvolvimento.
