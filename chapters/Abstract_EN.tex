\selectlanguage{brazil}%
GARAVITO, J. F. \foreignlanguage{english}{Ontologies and DSLs in the
generation of decision support systems, SustenAgro study case. Master
dissertation (Master Program in Computer Science And Computational
Mathematics.)} – Instituto de Ciências Matemáticas e de Computação,
Universidade de São Paulo, São Carlos – SP, 2017.

\vphantom{}

\selectlanguage{english}%
Decision Support Systems (DSSs) organize and process data and information
to generate results to support decision making in a specific domain.
They integrate knowledge from domain experts in each of their components:
models, data, mathematical operations (that process the data) and
analysis results. In traditional development methodologies, this knowledge
must be interpreted and used by software developers to implement DSSs.
That is because domain experts cannot formalize this knowledge in
a computable model that can be integrated into DSSs. The knowledge
modeling process is carried out, in practice, by the developers, biasing
domain knowledge and hindering the agile development of DSSs (as domain
experts cannot modify code directly). To solve this problem, a method
and web tool is proposed that uses ontologies, in the Web Ontology
Language (OWL), to represent expert's knowledge, and a Domain Specific
Language (DSL), to model DSS behavior. Ontologies, in OWL, are a computable
knowledge representations, which allow the definition of DSSs in a
format understandable and accessible to humans and machines. This
method was used to create the Decisioner Framework for the instantiation
of DSSs. Decisioner automatically generates DSSs from an ontology
and a description in its DSL, including the DSS interface (using a
Web Components library). An online ontology editor, using a simplified
format, allows that domain experts change the ontology and immediately
see the consequences of their changes in the in the DSS. A validation
of this method was done through the instantiation of the SustenAgro
DSS, using the Decisioner Framework. The SustenAgro DSS evaluates
the sustainability of sugarcane production systems in the center-south
region of Brazil. Evaluations, done by by sustainability experts from
Embrapa Environment (partners in this project), showed that domain
experts are capable of changing the ontology and DSL program used,
without the help of software developers, and that the system produced
correct sustainability analysis. 

\selectlanguage{brazil}%
\vphantom{}

\selectlanguage{english}%
\textbf{Keyworks:} \emph{Ontologies, Domain-Specific Language, Semantic
Web, Knowledge Representation, Decisioner Framework, Decision Support
System, SustenAgro }
