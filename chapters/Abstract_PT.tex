Os Sistemas de Apoio à Decisão (SAD\nomenclature{SAD}{Sistemas de Apoio à Decisão})
integram conhecimento dos especialistas do domínio em cada um dos
componentes deles: nos dados e modelos, nas operações matemáticas
que processam esses dados e nas informações resultantes que suportam
o processo de decisão. Nas metodologias de desenvolvimento tradicionais,
este conhecimento deve-se interpretar e implementar pelos desenvolvedores
de software, devido a que os especialistas não conseguem formalizar
esse conhecimento em um modelo computável para integrá-lo nos SADs.
O processo de modelagem de conhecimento é realizado pelos desenvolvedores,
parcializando o conhecimento do domínio e dificultando o desenvolvimento
ágil dos SADs. Dada está situação, identificou-se que não existe uma
representação de conhecimento computável que permita definir SADs,
que tenha um formato entendível e accessível pelos especialista do
domínio e pelas computadoras. A partir desse problema, foram testadas
soluciones, e encontrou-se que as ontologias baseadas na web semântica
representam conhecimento complexo e fornecem um formato entendível
pelos humanos e maquinas. O método Decisioner que permite aos especialistas
representar o conhecimento deles por meio de ontologias e \foreignlanguage{english}{Domain
Specific Language} (DSL\nomenclature{DSL}{Domain Specific Language}),
para permitir a definição e geração de SADs. O \foreignlanguage{english}{Framework}
Decisioner, implementar este método, fornecendo interfaces web de
edição da ontologia e da DSL para definir e gerar os SADs em entornos
web. Para validar dito método, o Framework Decisioner foi instanciado
com uma ontologia de avaliação da sustentabilidade em cana-de-açúcar
no centro-sul do Brasil e com uma descrição na DSL dos componentes
do SAD, permitiu a geração o SAD SustenAgro. Finalmente foram realizadas
avaliações no Framework Decisioner e no SAD SustenAgro para validar
o correto funcionamento deles.

\textbf{Palavras Chave:} \emph{Framework Decisioner, SAD, SustenAgro,
ontologias, web semântica, DSL, conhecimento computável.}
