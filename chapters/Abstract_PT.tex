GARAVITO, J. F. Ontologias e DSLs na geração de sistemas de apoio
à decisão, caso de estudo SustenAgro. Dissertação (Mestrado em Ciências
– Ciências de Computação e Matemática Computacional) – Instituto de
Ciências Matemáticas e de Computação, Universidade de São Paulo, São
Carlos – SP, 2017.

\vphantom{}

Os Sistemas de Apoio à Decisão (SAD\nomenclature{SAD}{Sistema de Apoio à Decisão})
organizam e processam dados e informações para gerar resultados que
apoiem a tomada de decisão em um domínio especifico. Eles integram
conhecimento de especialistas de domínio em cada um de seus componentes:
modelos, dados, operações matemáticas (que processam os dados) e resultado
de análises. Nas metodologias de desenvolvimento tradicionais, esse
conhecimento deve ser interpretado e usado por desenvolvedores de
software para implementar os SADs. Isso porque especialistas de domínio
não conseguem formalizar esse conhecimento em um modelo computável
que possa ser integrado aos SADs. O processo de modelagem de conhecimento
é realizado, na prática, pelos desenvolvedores, parcializando o conhecimento
do domínio e dificultando o desenvolvimento ágil dos SADs (já que
os especialistas não modificam o código diretamente). Para solucionar
esse problema, propõe-se um método e ferramenta web que usa ontologias,
na Web Ontology Language (OWL), para representar o conhecimento de
especialistas, e uma Domain Specific Language (DSL\nomenclature{DSL}{Domain Specific Language}),
para modelar o comportamento dos SADs. Ontologias, em OWL, são uma
representação de conhecimento computável, que permite definir SADs
em um formato entendível e accessível a humanos e máquinas. Esse método
foi usado para criar o framework Decisioner para a instanciação de
SADs. O Decisioner gera automaticamente SADs a partir de uma ontologia
e uma descrição na  DSL, incluindo a interface do SAD (usando uma
biblioteca de Web Components). Um editor online de ontologias, que
usa um formato simplificado, permite que especialistas de domínio
possam modificar aspectos da ontologia e imediatamente ver as consequência
de suas mudanças no SAD. Uma validação desse método foi realizada,
por meio da instanciação do SAD SustenAgro no Framework Decisioner.
O SAD SustenAgro avalia a sustentabilidade de sistemas produtivos
de cana-de-açúcar na região centro-sul do Brasil. Avaliações, conduzidas
por especialistas em sustentabilidade da Embrapa Meio ambiente (parceiros
neste projeto), mostraram que especialistas são capazes de alterar
a ontologia e DSL usadas, sem a ajuda de programadores, e que o sistema
produz análises de sustentabilidade corretas. 

\vphantom{}

\textbf{Palavras Chave:} \emph{Ontologias, Linguagem de Domínio Específico,
Web Semântica, Representação de Conhecimento, Framework Decisioner,
Sistema de apoio à decisão, SustenAgro}
