GARAVITO, J. F. Ontologias e DSLs na geração de sistemas de apoio
à decisão, caso de estudo SustenAgro. Dissertação (Mestrado em Ciências
– Ciências de Computação e Matemática Computacional) – Instituto de
Ciências Matemáticas e de Computação, Universidade de São Paulo, São
Carlos – SP, 2017.

\vphantom{}

Os Sistemas de Apoio à Decisão (SAD\nomenclature{SAD}{Sistemas de Apoio à Decisão})
integram conhecimento dos especialistas do domínio em cada um dos
componentes deles: nos dados e modelos, nas operações matemáticas
que processam esses dados e nas informações resultantes que suportam
o processo de decisão. Nas metodologias de desenvolvimento tradicionais,
este conhecimento deve-se interpretar e implementar pelos desenvolvedores
de software, devido a que os especialistas não conseguem formalizar
esse conhecimento em um modelo computável para integrá-lo nos SADs.
Sendo realizado o processo de modelagem de conhecimento pelos desenvolvedores,
parcializando o conhecimento do domínio e dificultando o desenvolvimento
ágil dos SADs. Dada esta situação, identificou-se que não existe uma
representação de conhecimento computável que permita definir SADs,
que tenha um formato entendível e accessível pelos especialista do
domínio e pelos computadores. A partir desse problema, foram testadas
soluções, e encontrou-se que as ontologias baseadas na web semântica
representam conhecimento complexo e fornecem um formato entendível
pelos humanos e máquinas. Com o qual propôs-se o método Decisioner
de consiste em que as ontologias complementadas com \foreignlanguage{english}{Domain
Specific Language} (DSL\nomenclature{DSL}{Domain Specific Language})
permitem aos especialistas representar o conhecimento deles. Este
método foi implementado no \foreignlanguage{english}{Framework} Decisioner
que permite definir e gerar SADs em entornos web. A validação do método
foi realizada por meio da instanciação do SAD SustenAgro no Framework
Decisioner, através de uma ontologia de avaliação da sustentabilidade
em cana-de-açúcar no centro-sul do Brasil e de uma descrição na DSL,
que permitiram gerar o SAD SustenAgro. Finalmente, foram realizadas
avaliações do Framework Decisioner e do SAD SustenAgro com resultados
positivos que permitem concluir que o método proposto forneceu um
meio de representação de conhecimento do domínio para que especialistas
possam definir SADs.

\vphantom{}

\textbf{Palavras Chave:} \emph{Ontologias, Linguagem de Domínio Específico,
Web Semântica, Representação de Conhecimento, Framework Decisioner,
Sistema de apoio à decisão, SustenAgro}
