Em desenvolvimento de software e engenharia de domínio uma linguagem
de domínio específico, em inglês \foreignlanguage{english}{\emph{Domain-Specific
Language}}\emph{ (DSL)}, é um tipo de linguagem de programação ou
linguagem de especificação, dedicada a um domínio particular de problema.

O conceito não é novo, linguagens de programação de propósito especifico
existiram desde o começo das linguagens de programação, mas o termo
tornou-se padrão devido à ascensão da modelagem de domínio específico.

Um usuário, relacionado com um domínio específico, pode usar uma DSL
sem ter experiência em desenvolvimento de software pois a DSL está
relacionada com seu domínio de trabalho. O autor \citet{fowler2010domain}
diz que programadores instruem o computador no que ele deve fazer,
pois já entendem a maneira dele trabalhar, mas com DSLs é feito o
inverso: o computador começa a entender o que o programador (usuário)
escreve.

No caso de uma arquitetura baseada em componentes para SADs, DSLs
podem ser criadas para domínios específicos de aplicação. Elas utilizariam
termos específicos do domínio e, assim, familiares a especialistas
desse domínio, com o qual seria possível a especialistas especificar
SADs com um grau de detalhamento grande o suficiente para permitir
a criação automática desses SADs, sem a necessidade da intervenção
de programadores. Os especialistas poderiam se tornar, na prática,
programadores de seus próprios SADs.

Segundo \citet{Mernik:2005:DDL:1118890.1118892} as vantagens das
DSL em comparação com as linguagens de proposito geral são a expressividade,
facilidade de uso e a integração com o domínio da aplicação

\section{Decisioner DSL}

No desenvolvimento da presente pesquisa foi necessário definir uma
DSL que permitisse representar as principais características do SAD
que precisávamos desenvolver, dito SAD foi desenhado para avaliar
a sustentabilidade em agricultura, pelo qual foram integrados conceitos
do domínio de conhecimento na definição dos componentes do SAD, fornecendo
uma linguagem para especialistas onde é suportada a de definição dos
SAD, as características da DSL são:

\subsection{Evaluation Object}

Os SAD focados na avaliação, é necessário definir um objeto de avaliação
que permita representar as entidades a avaliar, este objeto pelo geral
tem propriedades que vão representar cada uns dos indivíduos a avaliar,
pelo qual foi definido o comando \textit{evaluationObject, }que define
a estrutura do objeto de avalização e vincula os controles visuais,
o comando tem como argumentos a URI da classe dos elementos que vão
ser avaliados e cada uma das propriedades relacionadas. No código
\ref{lis:DSL-para-defini=0000E7=0000E3o} apresenta-se uma parte da
DSL que define a classe do objeto de avaliação \textit{ProductionUnit}
e as propriedades por meio dos comandos \foreignlanguage{english}{\textit{instance}}\textit{
}e\textit{ }\foreignlanguage{english}{\textit{type}}\textit{.}

\inputencoding{latin9}\begin{lstlisting}[caption={DSL: defini��o de Evaluation Object  },label={lis:DSL-para-defini=0000E7=0000E3o}]
evaluationObject ":ProductionUnit", {     
 instance "ui:hasName', label: ["en": "Name", "pt": "Nome"]
 instance ":hasAgriculturalProductionSystem"
 type label: ["en": "Type", "pt": "Tipo"]
}
\end{lstlisting}
\inputencoding{utf8}
O comando \foreignlanguage{english}{instance} vincula uma propriedade
definida na ontologia através da URI a qual pode estar complementada
por parâmetros que customizam a representação visual da propriedade.

O comando type vincula as subclasses da classe principal, para ser
atribuida nas intancias de Evaluation Object, no caso do Sistema SustenAgro,
dito comando identifica que as instancias de\textit{ ProductonUnit}
também podem ser um \textit{Provider} ou uma \textit{Plant}. Os parâmetros
que podem complementar os anteriores comandos são:
\begin{enumerate}
\item \textit{label}: define um texto associado
\item \textit{placeholder:} define um texto de ajuda
\item \textit{required}: define uma propriedade obrigatória
\item \textit{widget}: define um controle gráfico de usuário
\end{enumerate}

\subsection{Feature: }

O comando \textit{Feature} define as características que serão apresentadas
durante a avaliação para serem instanciadas como parte da Analysis,
ele tem como argumento uma URI que permite vincular as subclasses
da classe referenciada, as instancias destas classes serão quantificadas
mediante o processo da avaliação no qual é realizado o preenchimento
da propriedade \textit{has value }que vincula cada Feature com um
Value para quantificá-lo. No sistema SustenAgro foram estabelecidas
as Features por meio das URIs das classes: EnvironmentalIndicator,
EconomicIndicator, SocialIndicator, ProductionEfficiencyFeature e
TechnologicalEfficiencyFeature. Além disso é possível acrescentar
a inserção de \foreignlanguage{english}{\textit{features}} novas na
interface gráfica de usuário a través do parâmetro \foreignlanguage{english}{\textit{extraFeatures}}.\inputencoding{latin9}
\begin{lstlisting}[caption={DSL: defini��o de Features}]
feature ':EnvironmentalIndicator', 'extraFeatures': true
\end{lstlisting}
\inputencoding{utf8}

\subsection{Logica de avaliação: }

O comando \textit{Report} define o tratamento quantitativo que vai
ser efetuado às \foreignlanguage{english}{\textit{Features}}, com
a finalidade de obter valores gerais ou padrões como resultado do
processo de avaliação, suportando a definição de operações logicas
e aritméticas existentes tanto das linguagens Java e Groovy, fornecendo
assim uma linguagem para edição do metodo de avaliação, permitindo
atualizar o metodo dinámicamente e em tempo de execução, ditos valores
gerais são apresentados diretamente ou por meio de \foreignlanguage{english}{\textit{widgets}}
que facilitem a representação e compreensão da avaliação do sistema.
No código seguinte apresenta-se a implementação da formula do Sistema
SustenAgro, criando variáveis resultado de operações aritméticas para
gerar resultados gerais, no caso do SustenAgro o código gera a variável
\foreignlanguage{english}{\textit{sustainability}} que representa
o índice de sustentabilidade, más pode ser definido qualquer método
computável.\inputencoding{latin9}
\begin{lstlisting}[caption={DSL: defini��o da logica de avalia��o.}]
report {     
 environment = weightedSum(data.':EnvironmentalIndicator')
 economic = weightedSum(data.':EconomicIndicator')
 social = weightedSum(data.':SocialIndicator')
 sustainability = (environment + social + economic)/3
}
\end{lstlisting}
\inputencoding{utf8}
O comando \foreignlanguage{english}{\textit{report}} também define
as \foreignlanguage{english}{\textit{widgets}} que conformam a parte
visual do \foreignlanguage{english}{\textit{report}}, o qual pode
usar as variáveis de resultado da logica de avaliação como entrada
das \foreignlanguage{english}{\textit{widgets}} para melhorar a representação
e facilitar a compreensão dos resultados. No código seguinte apresenta-se
um exemplo de uso desta funcionalidade no sistema SustenAgro, no qual
são definidos comandos que geram as interfaces gráficas, como \textit{sustainabilityMatrix}
que usa as variáveis geradas anteriormente como argumentos.\inputencoding{latin9}
\begin{lstlisting}[caption={DSL: defini��o dos controles visuais do report}]
report {
 evaluationObjectInfo()
 sustainabilityMatrix x: sustainability, y: efficiency
 text 'en': 'Microregion map', 'pt': 'Mapa da microregi�o'
 map data.'Microregion'
}
\end{lstlisting}
\inputencoding{utf8} Por meio dessas configurações da DSL definiu-se as interfaces gráficas
de usuário do sistema para suportar o processo de avaliação, gerando
a representação visual dos Evaluation Objects, das Features, da logica
da avaliação e da interface gráfica do report.

Esta DSL permitirá que a interface gráfica seja definida em uma linguagem
de alto nível. Ela está baseada nas duas ontologias base e permite
definir e administrar os seguintes elementos conceituais:
\begin{itemize}
\item Indicadores
\item Componentes dos indicadores
\item Limiares
\item Métodos
\item Avaliações
\item Índices
\end{itemize}
Os elementos que compõem a DSL tem controles gráficos predefinidos
e será possível parametrizar as características destes controles gráficos
visuais. Por exemplo para as propriedades de tipo numérico contínuo
tem uma \foreignlanguage{english}{\textit{widget}} que representa
os valores reais que posem ser atribuídos em aquela propriedade, dita
\foreignlanguage{english}{\textit{widget}} pode ser mudada a outra
de acordo com as preferencias dos usuários. No caso das mudanças no
design são feitas através da edição do CSS3.
