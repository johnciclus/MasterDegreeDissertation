\usepackage{indentfirst}
\usepackage{float}
\usepackage{titlesec}
\usepackage{tikz, blindtext}
\usetikzlibrary{shapes.multipart}
\usepackage{calc,soul,fourier}
\usepackage{color, colortbl}
\usepackage{graphicx}
%\usepackage[brazil]{babel}
\usepackage[brazilian,hyperpageref]{backref}
\setlength{\parindent}{48pt}
\usepackage[alf]{abntex2cite}
%\usepackage{lmodern}
\usepackage[unicode=true]{hyperref}
\hypersetup{%
pdfborder = {0 0 0},
colorlinks ,
citecolor=blue,
filecolor=blue,
linkcolor=blue,
urlcolor=blue
}


%================================================================================
%======================================
%estilo do section, subsection e subsubsection
%======================================
 
\titleformat{\section}
{\normalfont\fontsize{16pt}{18pt}\selectfont\bfseries}{\thesection}{1em}{}
\titleformat{\subsection}
{\normalfont\fontsize{14pt}{16pt}\selectfont\bfseries}{\thesubsection}{1em}{}
\titleformat{\subsubsection}
{\normalfont\fontsize{12pt}{14pt}\selectfont\bfseries}{\thesubsubsection}{1em}{}


%================================================================================
%======================================
%Pacotesdecitações
%======================================
\renewcommand*{\backref}[1]{}
\renewcommand*{\backrefalt}[4]{
\ifcase #1 (No citado.)
\or (Citado na página~#2.)
\else (Citado nas páginas #2.)
\fi%
}
\renewcommand*{\backrefsep}{, }
\renewcommand*{\backreftwosep}{e~}
\renewcommand*{\backreflastsep}{e~}
%================================================================================
% Configuração dos Estilo dos Capítulo
%================================================================================

\makechapterstyle{box}{

\renewcommand{\chapterheadstart}{}
% Secao secundaria (Section) Caixa baixa, Negrito
\renewcommand*{\cftsectionfont}{\bfseries}
% Secao terciaria (Subsection) Caixa baixa, Negrito, italico
\renewcommand*{\cftsubsectionfont}{\itshape\bfseries}
% Secao quaternaria (Subsubsection) Caixa baixa, italico
\renewcommand*{\cftsubsubsectionfont}{\itshape}
% Secao quinquenária (Subsubsubsection) Caixa baixa
\renewcommand*{\cftparagraphfont}{\normalsize}

% tamanhos de fontes de chapter e part
\ifthenelse{\equal{\ABNTEXisarticle}{true}}{%
\setlength\beforechapskip{\baselineskip}
\renewcommand{\chaptitlefont}{\ABNTEXsectionfont\ABNTEXsectionfontsize}
}{%else
\setlength{\beforechapskip}{0pt}
\renewcommand{\ABNTEXchapterfontsize}{\LARGE}
%\renewcommand{\ABNTEXchapterfont}{\sffamily\bfseries}
%alteração da fonte dos capítulos, seções e subseções
\renewcommand{\chaptitlefont}{\ABNTEXchapterfont\bfseries\ABNTEXchapterfontsize}
}
%\renewcommand{\chapter}{\chaptertitlename\ \thechapter}{0pt}{\Large\uppercase}
\renewcommand{\chapnumfont}{\chaptitlefont}
\renewcommand{\parttitlefont}{\ABNTEXpartfont\ABNTEXpartfontsize}
\renewcommand{\partnumfont}{\ABNTEXpartfont\ABNTEXpartfontsize}
\renewcommand{\partnamefont}{\ABNTEXpartfont\ABNTEXpartfontsize}

\renewcommand*{\printchaptername}{}
\renewcommand*{\chapnumfont}{\normalfont\sffamily\huge\bfseries}
\renewcommand*{\printchapternum}{
\hrulefill{\renewcommand{\arraystretch}{1.5}
\begin{tabular}{|c|}
\rowcolor{black}\color{white}\normalsize\ABNTEXchapterfont\MakeTextUppercase{\chaptername}\\
\vspace{-1.5ex}\\
\resizebox{!}{1.1cm}{\ABNTEXchapterfont\thechapter}
\\[2.5ex]
\hline
\end{tabular}
}
}

\renewcommand*{\chaptitlefont}{\normalfont\sffamily\Huge\bfseries}
\renewcommand*{\printchaptertitle}[1]{
\flushright\chaptitlefont##1
\vskip -0.6ex\hfill\rule{.8\textwidth}{0.5pt} \\
\vskip -2.8ex\hfill\rule{.8\textwidth}{2pt}\\
\vskip 1.5ex
}

}

\chapterstyle{box}

%=================================
%Configurações do estilo da pagina
%==================================

%\makepagestyle{ruled}
%\makeoddfoot{ruled}{}{}{}
%\makeevenfoot{ruled}{}{}{}
%\makeheadrule{ruled}{\textwidth}{\normalrulethickness}
%\makeevenhead{ruled}{\thepage}{}{\small\itshape\leftmark}
%\makeoddhead{ruled}{\small\itshape\rightmark}{}{\thepage}
%\makeatletter % because of \@chapapp
%\makepsmarks  {ruled}{
%\nouppercaseheads
%\createmark	{chapter} 	{both} {shownumber}{\@chapapp\ }{. \ }
%\createmark	{section} 	{right} {shownumber}{}		  {. \ }
%\createplainmark {toc}		{both}{\contentsname}
%\createplainmark {bib}		{both}{\bibname}
%}
%\makeatother
%\pagestyle{ruled}

%================================================================================
% Configuração das legendas nas figuras e tabelas
%================================================================================

\newcommand{\fautor}{\legend{Fonte: Elaborada pelo autor.}}
\newcommand{\fadaptada}[2][]{\legend{Fonte: Adaptada de \citeonline[#1]{#2}.}}
\newcommand{\fdireta}[2][]{\legend{Fonte: \citeonline[#1]{#2}.}}
\newcommand{\fdadospesquisa}{\legend{Fonte: Dados da pesquisa.}}
